\documentclass[italian,a4paper]{article}
\usepackage{babel,amsmath,amssymb,amsthm,url}
\usepackage[text={6in,9in},centering]{geometry}
\usepackage[utf8x]{inputenc}
\usepackage[T1]{fontenc}
\usepackage{ae,aecompl}
\frenchspacing
\pagestyle{empty}
%------------- eliminare prime e ultime linee isolate
\clubpenalty=9999%
\widowpenalty=9999
%------------- impostazioni environment teoremi
\theoremstyle{definition}
\newtheorem{art}{Art.}
%
%------------- ridefinizione simbolo per elenchi puntati: en dash
\renewcommand{\labelenumi}{\arabic{enumi}.}
\renewcommand{\labelitemi}{--}
%------------- nuovi environment senza spazi
%\newenvironment{packeditem}{
%\begin{itemize}
%  \setlength{\itemsep}{1pt}
%  \setlength{\parskip}{0pt}
%  \setlength{\parsep}{0pt}
%}{\end{itemize}}
%\newenvironment{packedenumeration}{
%\begin{enumerate}
%  \setlength{\itemsep}{1pt}
%  \setlength{\parskip}{0pt}
%  \setlength{\parsep}{0pt}
%}{\end{enumerate}}
%\newenvironment{packeddescription}{
%\begin{enumerate}
%   \setlength{\itemsep}{1pt}
%   \setlength{\parskip}{0pt}
%   \setlength{\parsep}{0pt}
%}{\end{enumerate}}
%--------- comandi insiemi numeri complessi, naturali, reali e altre abbreviazioni
\newcommand{\eqnum}{\setcounter{equation}{0}}
%---------
%---------

\title{Codice del tressette}
\author{\url{www.fitres.it}}
\date{}

\begin{document}
\maketitle
\section*{Le dichiarazioni}

\setcounter{art}{35}
\begin{art}
Il diritto di eseguire una o più dichiarazioni espresse spetta solo al
giocatore di mano, al quale non è però consentito di eseguire o completare
alcuna dichiarazione dopo che abbia eseguito la giocata di battuta.
\end{art}

\begin{art}
Ciascuna dichiarazione può riferirsi: esplicitamente, al palo di pigliata o
al palo al quale ci si accinge a giocare;  genericamente, ad uno o più pali.
\end{art}

\subsection*{Dichiarazioni relative al palo di pigliata}

\begin{art}
Con riferimento al palo di pigliata, le dichiarazioni consentite possono
essere:
\begin{itemize}
    \item \emph{(ho fatto) questo piombo}, quando la presa sia stata effettuata
con l’unica o con l’ultima carta del palo in suo possesso;
    \item \emph{(sono) liscio qui}, quando ancora si possegga almeno una carta
soggetta rispetto alle carte possedute al palo dagli altri giocatori.
\end{itemize}
\end{art}

\subsection*{Dichiarazioni relative a pali diversi da quello di pigliata e di giocata}

\begin{art}
Con riferimento ai pali diversi da quello di pigliata e da quello al quale
ci si accinge a giocare, le dichiarazioni consentite sono:
\begin{itemize}
    \item \emph{feci un piombo}, per comunicare di aver esaurito le carte ad uno o
più pali diversi da quelli ai quali si sia giocato prima della battuta nella
quale il giocatore ha rilevato la mano.

   \item \emph{(sono) faglio a un palo}, per comunicare che, in origine o
(indistintamente) al momento della dichiarazione, si era o si è sprovvisti
di carte ad uno o più pali.

\item       \emph{faglio al (palo del) mio compagno}, per comunicare di non
possedere alcuna carta al palo dove si ritiene che il compagno, attraverso
accuse, sfide, o scarti dello stesso o per altre circostanze di gioco, abbia
interesse; tale dichiarazione non è consentita se al palo si posseggono solo
carte sovrane.

\item       \emph{liscio al (palo del)  mio compagno}, per comunicare il possesso di
almeno una carta non sovrana al palo dove si ritiene che il compagno,
attraverso accuse, sfide o scarti dello stesso o per altre circostanze di
gioco, abbia interesse.
\end{itemize}
\end{art}

\begin{art}
L’erronea dichiarazione di faglio o di liscio al palo del compagno eseguita
in buona fede non è sanzionabile.
\end{art}

\subsection*{Dichiarazioni relative al palo di giocata}

\begin{art}
Con riferimento al palo al quale ci si accinge a giocare, è consentito
dichiarare:


\begin{itemize}
\item       il possesso di una Napoletana;
\item       il possesso di un gruppo di due carte di tressette;
\item       il possesso di una singola carta di tressette;
\item       il possesso del Re;
\item       l’eventuale grado sovrano della carta che si gioca;
\item         la semplice lunghezza del palo.
\end{itemize}
\end{art}

\subsection*{La dichiarazione della Napoletana}

\begin{art}
Il giocatore che intende dichiarare il possesso di una Napoletana al palo di
giocata può farlo espressamente, giocando una qualsiasi carta e precisando
la lunghezza del palo.
\end{art}

\subsection*{Le sfide}

\begin{art}
Si definisce sfida una dichiarazione relativa ad un palo con due carte di
tressette o ad un palo almeno terzo con una sola carta di tressette.
\end{art}

\begin{art}
Se si possiede una sola di carta di tressette di un palo almeno terzo, la
sfida può essere lanciata giocando qualsiasi carta semplice.
\end{art}

\begin{art}
Se si posseggono due carte di tressette (che vanno entrambe dichiarate):

\begin{itemize}
\item  se il palo è secondo, la dichiarazione comporta l’obbligo di giocare
quella superiore, salvo che il palo sia costituito dal Venticinque, nel qual
caso a tale obbligo si sostituisce quello di dover giocare le due carte
l’una di seguito all’altra;

\item  se il palo dichiarato è terzo, resta solo esclusa la possibilità di
giocare una carta semplice;

\item   se il palo è più che terzo, si può eseguire la dichiarazione giocando
qualsiasi carta. 
\end{itemize}
\end{art}

\begin{art}
Nelle sfide di Ventinove, di Ventotto e di Venticinque non è mai consentito
dichiarare anche l’eventuale possesso del Re che non venga giocato.
\end{art}

\subsection*{La dichiarazione del Re}

\begin{art}
La dichiarazione relativa al possesso del Re, senza giocarlo, è consentita
solo giocando una carta semplice di un palo almeno terzo senza carte di
tressette.
\end{art}

\subsection*{La dichiarazione di sovranità della carta giocata}

\begin{art}
La dichiarazione di sovranità della carta giocata può riguardare sia una
carta diventata sovrana in assoluto sia una carta diventata tale perch\'e
quella o quelle effettivamente sovrane sono possedute anch’esse dal
giocatore e non sono n\'e carta di tressette n\'e il Re.
\end{art}

\begin{art}
L’erronea dichiarazione di sovranità eseguita in buona fede non è
sanzionabile salva l’ipotesi che la sovranità della carta dipenda dal
possesso di una carta di tressette o del Re.
\end{art}

\begin{art}
Chi esegue la dichiarazione di sovranità deve sempre precisare la lunghezza
del palo.
\end{art}

\subsection*{La semplice dichiarazione di lunghezza del palo}

\begin{art}
La semplice dichiarazione di lunghezza del palo non può essere eseguita se
tra le carte non giocate ve ne sia una di tressette o il Re.
\end{art}

\begin{art}
Se si dichiara un palo secondo è obbligatorio giocare la carta di grado
superiore, anche se le due carte sono consecutive di grado.
\end{art}

\begin{art}
Se si vuol rendere noto lo stato di unica o di ultima carta del palo a cui
si gioca,  è sufficiente dichiarare \emph{sola}, sia che si tratti di carta
soggetta, sia che si tratti di carta sovrana; nel secondo caso, però, è
possibile dichiarare \emph{piombo}.

Al riguardo si richiama e si conferma la previsione dell’art. 49.
\end{art}

\subsection*{Altre norme concernenti le dichiarazioni relative al palo di giocata}

\begin{art}
Se una qualsiasi dichiarazione viene eseguita senza precisare la lunghezza
del palo, si sottintende che il palo sia esattamente terzo; se il palo è
quarto o più che quarto, occorre precisare ulteriormente la dichiarazione
aggiungendo alla stessa, rispettivamente: \emph{lungo} o  \emph{lunghissimo}.
\end{art}

\begin{art}
Il giocatore che non reputi opportuno condizionare la propria giocata
all’integrale rispetto delle norme contenute nei precedenti articoli, deve
rinunciare ad eseguire la dichiarazione relativa al palo di giocata.
\end{art}

\begin{art}
Nel lanciare una sfida e, più in generale, nell’eseguire una qualsiasi
dichiarazione, occorre usare le espressioni, anche di tipo gestuale, in uso
al tavolo.
\end{art}

\begin{art}
Se al tavolo si utilizzano, per la stessa fattispecie dichiarativa, più
espressioni alternative, è corretto che ciascun giocatore adoperi sempre la
stessa espressione; ciò al fine di non generare il dubbio che a
dichiarazioni variamente formulate corrispondano fattispecie secondarie
diverse che non possono essere palesate.
\end{art}

\begin{art}
Altre norme riportate nelle successive parti del codice prevedono
limitazioni alla possibilità di eseguire le precedenti dichiarazioni.
\end{art}

\section*{Gioco ordinario: norme moderne}
\subsection*{Norme che regolano le dichiarazioni}
\setcounter{art}{145}
\begin{art}
S’intende qui confermato e richiamato l’intero contenuto degli articoli da
36 a 57 del presente Codice, fatta esclusione per quelle norme eventualmente
in contrasto con quanto ulteriormente o diversamente sancito negli articoli
che seguono.
\end{art}

\begin{art}
Il giocatore che inizia tacitamente una battuta con una carta (semplice o di
Tressette) piombo, sola, lisciante, terza, lunga o lunghissima non può
palesare neppure parzialmente lo stato del palo dopo che abbia eventualmente
conquistata la pigliata, e cioè non può dire, accingendosi a giocare ad
altro palo: ho fatto questo piombo oppure sono liscio qui.
\end{art}

\begin{art}
Può però, avendo conseguito la pigliata su una sua giocata tacita, rigiocare
– anche di seguito - allo stesso palo ancora tacitamente o anche utilizzando
una delle formule di cui al 
\end{art}

\begin{art}
E potrà anche (riprendendo successivamente la mano su una giocata da altri
iniziata, anche ad un diverso palo) regolarmente riferirsi al palo al quale
aveva giocato in silenzio utilizzando una delle formule di cui al
\end{art}

\begin{art}
Le dichiarazioni possibili con riferimento al palo di giocata sono le
seguenti:

\begin{enumerate}
    \item        \emph{sola} giocando l’unica o l’ultima carta del palo, purch non
sovrana

\item        \emph{piombo} giocando l’unica o l’unica carta del palo se trattasi di
carta (anche notoriamente) sovrana

\item        \emph{si liscia} giocando la carta di tressette (sovrana o soggetta) o,
comunque, la carta di grado superiore (sovrana o soggetta) ad un palo ove è
presente solo un’altra carta semplice

\item        \emph{è mia e si liscia} giocando la carta sovrana di tressette o,
comunque, la carta sovrana di grado superiore ad un palo ove è presente solo
un’altra carta semplice

\item        \emph{terzo} giocando una carta sovrana o soggetta diversa dal Re ad un
palo ove siano presenti altre due carte semplici tra cui può esserci o non
esserci il Re

\item       \emph{si liscia due volte} giocando la carta di tressette (sovrana o
soggetta) o, comunque, la carta di grado superiore (sovrana o soggetta) ad
un palo ove sono  presenti altre due carte semplici

\item        \emph{è mia e si liscia due volte} giocando la carta sovrana di
tressette o, comunque, la carta sovrana di grado superiore ad un palo ove
sono presenti altre due carte semplici

\item       \emph{è mia, terzo} giocando una carta di tressette sovrana o una carta
semplice sovrana o soggetta esclusivamente ad una o ad entrambe le altre due
carte semplici possedute dal giocatore ad un palo terzo 

\item        \emph{vero doppio liscio} o \emph{terzo di Re} giocando una qualsiasi carta
semplice di un palo terzo senza carte di tressette con il Re che non viene
giocato

\item      \emph{lungo} giocando una carta sovrana o soggetta diversa dal Re ad un
palo ove siano presenti altre tre carte semplici tra cui può esserci o non
esserci il Re

\item     \emph{si liscia, lungo} giocando la carta di tressette (sovrana o
soggetta) o, comunque, la carta di grado superiore (sovrana o soggetta) ad
un palo ove sono  presenti altre tre carte semplici

\item      \emph{è mia, lungo} giocando una carta di tressette sovrana o una carta
semplice sovrana o soggetta esclusivamente ad una o a più carte semplici
possedute dal giocatore ad un palo quarto 

\item      \emph{lungo tutta la Napoletana} o \emph{lungo di Re} giocando una qualsiasi
carta semplice di un palo quarto senza carte di tressette con il Re che non
viene giocato

\item      \emph{lunghissimo} giocando una carta sovrana o soggetta diversa dal Re
ad un palo ove siano presenti almeno altre quattro carte semplici tra cui
può esserci o non esserci il Re

\item      \emph{si liscia, lunghissimo} giocando la carta di tressette (sovrana o
soggetta) o, comunque, la carta di grado superiore (sovrana o soggetta) ad
un palo ove sono  presenti almeno altre quattro carte semplici

\item      \emph{è mia, lunghissimo} giocando una carta di tressette sovrana o una
carta semplice sovrana o soggetta esclusivamente ad una o a più carte
semplici possedute dal giocatore ad un palo quinto

\item      \emph{lunghissimo tutta la Napoletana} o \emph{lunghissimo di Re} giocando una
qualsiasi carta semplice di un palo almeno quinto senza carte di tressette
con il Re che non viene giocato

\item      \emph{liscio e busso} o \emph{Asso terzo} giocando una qualsiasi carta semplice
di un palo terzo con l’Asso

\item      \emph{liscio e busso lungo} o \emph{Asso lungo} giocando una qualsiasi carta
semplice di un palo quarto con l’Asso

\item      \emph{liscio  busso lunghissimo} o \emph{Asso lunghissimo} giocando una
qualsiasi carta semplice di un palo almeno quinto con l’Asso

\item      \emph{ribusso} o \emph{Due terzo} giocando una qualsiasi carta semplice di un
palo terzo con il Due

\item      \emph{ribusso lungo} o \emph{Due lungo} giocando una qualsiasi carta semplice
di un palo quarto con il Due

\item      \emph{ribusso lunghissimo} o \emph{Due lunghissimo} giocando una qualsiasi
carta semplice di un palo almeno quinto con il Due

\item      \emph{busso} o \emph{Tre terzo} giocando una qualsiasi carta semplice di un
palo terzo con il Tre

\item      \emph{busso lungo} o \emph{Tre lungo} giocando una qualsiasi carta semplice di
un palo quarto con il Tre

\item      \emph{busso lunghissimo} o \emph{Tre lunghissimo} giocando una qualsiasi carta
semplice di un palo almeno quinto con il Tre

\item      \emph{liscio} o \emph{la migliore} o \emph{ventotto piombo} giocando il due di un
ventotto piombo

\item      \emph{doppio liscio} o \emph{la migliore} o \emph{ventotto terzo} giocando il Due o
l’Asso di un ventotto terzo

\item      \emph{la migliore, lungo} o \emph{ventotto lungo} giocando una qualsiasi carta
di un ventotto lungo

\item      \emph{la migliore, lunghissimo} o \emph{ventotto lunghissimo} giocando una
qualsiasi carta di un ventotto lunghissimo

\item      \emph{liscio} o \emph{la buona} o \emph{ventinove piombo} giocando il Tre di un
ventinove piombo

\item      \emph{doppio liscio} o \emph{la buona} o \emph{ventinove terzo} giocando il Tre o
l’Asso di un ventinove terzo

\item      \emph{la buona, lungo} o \emph{ventinove lungo} giocando una qualsiasi carta
del ventinove lungo

\item      \emph{la buona, lunghissimo} o \emph{ventinove lunghissimo} giocando una
qualsiasi carta di un ventinove lunghissimo

\item      \emph{è mio liscio e piombo} o \emph{venticinque piombo} giocando
consecutivamente il Due ed il Tre di un venticinque piombo

\item      \emph{è mio terzo} o \emph{venticinque terzo} giocando il Due di un venticinque
terzo con o senza il Re

\item      \emph{è mio e voglio l’Asso} o \emph{venticinque terzo di Re} giocando il Due
di un venticinque terzo con il Re

\item      \emph{è mio, lungo} o \emph{venticinque lungo} giocando una qualsiasi carta di
un venticinque lungo con o senza il Re

\item      \emph{è mio, lungo e voglio l’Asso} o \emph{venticinque lungo di Re} giocando
una qualsiasi carta diversa dal Re di un venticinque lungo con il Re

\item      \emph{è mio, lunghissimo} o \emph{venticinque lunghissimo} giocando una
qualsiasi carta di un venticinque lunghissimo con o senza il Re

\item      \emph{è mio, lunghissimo e voglio l’Asso} o \emph{venticinque
    lunghissimo di Re} giocando una qualsiasi carta diversa dal Re di un venticinque lunghissimo
con il Re.
\end{enumerate}
\end{art}
\end{document}
