\documentclass[italian,a4paper]{book}
\usepackage{babel,amsmath,amssymb,amsthm,url}
\usepackage[text={6in,9in},centering]{geometry}
\usepackage[utf8x]{inputenc}
\usepackage[T1]{fontenc}
\usepackage{ae,aecompl}
\frenchspacing
\pagestyle{plain}
%------------- eliminare prime e ultime linee isolate
\clubpenalty=9999%
\widowpenalty=9999
%------------- impostazioni environment teoremi
\theoremstyle{definition}
\newtheorem{art}{Art.}
%
%------------- ridefinizione simbolo per elenchi puntati: en dash
\renewcommand{\labelenumi}{\arabic{enumi}.}
\renewcommand{\labelitemi}{--}
%------------- nuovi environment senza spazi
\newenvironment{packeditem}{
\begin{itemize}
  \setlength{\itemsep}{1pt}
  \setlength{\parskip}{0pt}
  \setlength{\parsep}{0pt}
}{\end{itemize}}
\newenvironment{packedenum}{
\begin{enumerate}
  \setlength{\itemsep}{1pt}
  \setlength{\parskip}{0pt}
  \setlength{\parsep}{0pt}
}{\end{enumerate}}
\newenvironment{packeddescription}{
\begin{description}
   \setlength{\itemsep}{1pt}
   \setlength{\parskip}{0pt}
   \setlength{\parsep}{0pt}
}{\end{description}}
%--------- comandi insiemi numeri complessi, naturali, reali e altre abbreviazioni
\newcommand{\eqnum}{\setcounter{equation}{0}}
%---------
%---------
\title{Regole del tressette classico}
\author{}
\date{}
\begin{document}
\maketitle
\chapter{Norme generali}
\section{Le carte}
\subsection{I tipi di carte}
\begin{art}
    Il tressette, i giochi affini e i giochi derivati si praticano preferibilmente con un mazzo di carte \emph{napoletane}; con l'accordo di tutti i giocatori è però possibile adoperare altri tipi di carte in uso nelle diverse regioni italiane o, anche, le carte francesi.
\end{art}
\begin{art} \hspace*{\fill}
    \begin{packeditem}
    \item Le carte sono quaranta, divise in quattro \emph{pali}: denari, spade, coppe e bastoni.
\item Tra i pali non esiste gerarchia ma solo, per determinate finalità di carattere tecnico o regolamentare, un ordine convenzionale che corrisponde a quello appena seguito.
\item Ciascuno dei suddetti pali si compone di dieci carte: l'Asso (ossia l'Uno), il Due, il Tre, il Quattro, il Cinque, il Sei, il Sette, l'Otto (rappresentato da una Donna o da un Fante e così denominato), il Nove (rappresentato da un Cavaliere a cavallo e più semplicemente denominato Cavallo) ed il Dieci (rappresentato da un Re e così denominato).
\item Volendo riferirsi a ciascuna carta in forma sintetica scritta, se ne
    indica il palo con la iniziale minuscola (d, s, c, b) seguito da un carattere alfanumerico: A, 2,
    3, 4, 5, 6, 7, D, C, R.
    \end{packeditem}
\end{art}
\begin{art}
    Nel caso si utilizzi un mazzo di carte francesi,
    \begin{packeditem}
\item   ai pali di denari, spade, coppe, bastoni delle carte napoletane corrispondono rispettivamente i seguenti semi: cuori, picche, quadri, fiori;
\item   le carte dall'Otto al Dieci devono essere eliminate ed il K, il Q ed il J corrispondono al Re, alla Donna ed al Cavallo delle carte napoletane.
    \end{packeditem}
\end{art}
\subsection{Gradi, valori e definizioni delle carte}
\begin{art}
    Il grado delle carte in ciascun palo per ordine decrescente è il seguente: il Tre, il Due, l'Asso, il Re, il Cavallo, la Donna, il Sette, il Sei, il Cinque ed il Quattro.
\end{art}
\begin{art}\label{carta.sovrana}
    Il Tre è la carta naturalmente \emph{sovrana} in ciascun palo; ciascuna
    delle altre può divenire tale se tutte le carte di grado superiore sono
    uscite; una carta non sovrana si definisce \emph{soggetta}. 
\end{art}
\begin{art}
    Il Tre, il Due e l'Asso sono definite \emph{carte di tressette}, e, in
    particolare, il Due e l'Asso sono definite carte secondarie di
    tressette; le altre sono definite carte semplici, distinguendosi a loro
    volta in \emph{figure} (Re, Donna e Cavallo) e \emph{cartine} (Sette,
    Sei, Cinque e Quattro). \`E possibile riferirsi al Tre anche con la
    definizione di \emph{la migliore}; altre definizioni previste sono:
    \emph{la buona}, riferendosi al Due, e \emph{le buone} riferendosi al Due e all'Asso.
\end{art}
\begin{art}
    Il gruppo delle tre Carte di Tressette di uno stesso palo costituisce
    una \emph{Napoletana}, quello delle tre figure una \emph{Terza Reale},
    quello formato dal Tre e dal Due un \emph{Venticinque}, quello dal Due e
    dell'Asso un \emph{Ventotto}, e quello dal Tre e dall'Asso un
    \emph{Ventinove}.
\end{art}
\begin{art}
    Ciascun Asso vale un punto; le altre venti carte costituite dalle otto Carte di Tressette e dalle dodici figure valgono ciascuna un terzo di punto; le frazioni di punto non si conteggiano.
\end{art}
\section{Il gioco}
\subsection{Il gioco della carta e la passata}
\begin{art}
    Lo scopo del gioco consiste nel cercare di conquistare il massimo degli undici punti di volta in volta disponibili, dei quali: dieci sono portati dai valori delle carte ed uno dal valore dell'ultima presa.
\end{art}
\begin{art}
    Gli undici punti sono definiti \emph{di
    mazzo} per distinguerli da quelli, eventuali, detti \emph{di buongioco} previsti solo nel gioco del tressette con l'accusa.
\end{art}
\begin{art}
    Il gioco si svolge in passate, una o più a seconda del punteggio fissato per la vittoria; tutte le passate si esauriscono con la giocata delle quaranta carte, tranne l'ultima che può finire anticipatamente per il raggiungimento del punteggio fissato per la vittoria.
\end{art}
\subsubsection{La passata: attività preliminari}
\begin{art}\label{passata.iniz}
    Propedeutiche allo svolgimento di ogni passata sono le seguenti operazioni: la mischiata, l'alzata e la distribuzione.
\end{art}
\begin{art}
    Le carte vanno mischiate prima della distribuzione; la mischiata è
    consentita a tutti i giocatori ma chi deve distribuirle, ossia il
    \emph{cartaro}, ha il diritto di mischiarle per ultimo.
\end{art}
\begin{art} \hspace*{\fill}
    \begin{packeditem}
\item Dopo che le carte sono state mischiate, il mazzo di carte deve essere diviso in due porzioni accidentali; questa operazione va fatta dal giocatore che si trova alla sinistra del distributore.
\item Si può convenire che l'alzata, quando il giocatore a sinistra del
    distributore sia compagno dello stesso, debba essere fatta
    dall'avversario di fronte; in tal modo si ottiene che ciascuna delle due
    operazioni --- cioè l'alzata e la distribuzione --- viene espletata da giocatori avversari.
\item L'alzata è sempre obbligatoria e, una volta effettuata, non è più consentito a nessun giocatore di rimischiare le carte. Nel fare l'alzata bisogna evitare che una delle due porzioni, in cui resta diviso il mazzo, contenga meno di cinque carte; altrimenti l'alzata deve essere ripetuta.
    \end{packeditem}
\end{art}
\begin{art}
    Il distributore, dopo l'alzata, deve riunire di nuovo i due monti in uno
    solo, mettendo di sopra quello che prima era di sotto; poi deve
    distribuire le carte, in senso antiorario, dalla parte superiore del monte.
\end{art}
\begin{art}
    Le carte vanno rimischiate e la distribuzione ripetuta:
    \begin{packeditem}
\item    qualora i giocatori non siano tutti presenti, poiché gli assenti verrebbero ad essere privati della facoltà di mischiare anch'essi le carte prima della distribuzione;
\item    qualora durante la distribuzione si scoprano due o più carte, o anche una sola se questa è \emph{di tressette}.
\item    qualora ne cada qualcuna sul pavimento, anche quando non si vede quale sia;
\item    qualora uno dei giocatori riceva un numero di carte diverso da quello previsto o, pur nel numero esatto, le riceva con modalità diverse da quelle previste.
\item      quando, a gioco in corso, venga riscontrato che un giocatore aveva ricevuto un numero di carte inferiore o superiore a quello previsto.
    \end{packeditem}
\end{art}
\subsubsection{La passata: svolgimento}
\begin{art} \hspace*{\fill}
    \begin{packeditem}
\item Ogni passata consta di tante battute quante sono le carte distribuite a ciascun giocatore.
\item Per \emph{battuta} s'intende il complesso della carta giocata
    inizialmente da un giocatore (\emph{carta di battuta}) e delle carte
    (\emph{di risposta}) giocate da ciascuno degli altri giocatori.
\item La battuta inizia ufficialmente con la giocata o con una qualsiasi dichiarazione eseguita dal giocatore di mano e cessa con il deposito della pigliata.
    \end{packeditem}
\end{art}
\begin{art}
    Una volta che la carta giocata ha toccato il tavolo o se, anche senza averlo toccato, è stata vista da uno o più giocatori, la stessa non può più essere ritirata. Può esserne però corretto, con immediatezza, l'eventuale errato palesamento.
\end{art}
\begin{art}   \hspace*{\fill}
    \begin{packeditem}
\item \`E vietato giocare, tranne il caso di messa a terra definitiva
    (disciplinato dagli articoli da~\ref{messa.terra} a~\ref{fine.terra}), contemporaneamente più di una carta.
\item Non è mai consentito evidenziare di aver già pronta la carta da giocare prima che sia giunto il proprio turno di gioco e, tanto meno, può essere consentito a chi abbia commesso tale infrazione di aggravare il proprio comportamento sostituendo tale carta alla luce delle carte nel frattempo giocate o delle dichiarazioni verbali eseguite dagli altri giocatori.
    \end{packeditem}
\end{art}
\begin{art}
    La giocata di battuta compete al giocatore di mano che, generalmente, è libero di giocare qualsiasi carta; a ciascuno degli altri compete la giocata di risposta che obbliga sempre il giocatore a giocare, se ne possiede, una carta dello stesso palo della carta della giocata di battuta; le giocate di risposta devono essere eseguite ordinatamente in senso antiorario.
\end{art}
\begin{art}
    Alla prima battuta è di mano il giocatore alla destra del cartaro; ad ogni battuta successiva è di mano il giocatore che ha conquistato la battuta precedente.
\end{art}
\begin{art}
    Prima di dare inizio alla battuta iniziale il giocatore di mano deve
    assicurarsi di aver ricevuto il giusto numero di carte e deve
    assicurarsi che la stessa verifica sia stata eseguita dagli altri
    giocatori; tali adempimenti rientrano nella fase della distribuzione per
    cui, ai fini del presente Codice, la passata ha inizio nel momento in
    cui un giocatore esegue una qualsiasi dichiarazione espressa o d\`a inizio alla prima battuta.
\end{art}
\begin{art}
    Prima di dare inizio a ciascuna battuta successiva a quella iniziale il giocatore di mano deve attendere che la battuta precedente sia stata interamente raccolta e correttamente depositata.
\end{art}
\begin{art}
    Ad ogni giocatore è consentito, prima che sia stata iniziata la battuta
    successiva, di rivedere la pigliata appena conclusasi (e non altre) da chiunque conseguita.
\end{art}
\begin{art}
    Le giocate di risposta allo stesso palo della giocata di battuta devono essere eseguite rigorosamente con immediatezza, quelle di risposta ad un palo diverso (scarti) devono essere eseguite, per correttezza, senza immotivati indugi.
\end{art}
\begin{art}
    La battuta è conquistata, o (come si dice più correttamente) è
    \emph{pigliata}, dal giocatore che ha giocato la carta di grado maggiore tra quelle giocate al palo della giocata di battuta.
\end{art}
\begin{art}
    Con la conquista della battuta il giocatore
    conquista anche le carte giocate, con i relativi valori.
\end{art}
\subsubsection{La passata: conclusione}
\begin{art}
    La passata si esaurisce con la giocata di tutte le carte ma si considera ufficialmente conclusa a tutti gli effetti solo dopo che sia stato accertato ed annotato il punteggio o, nel caso la partita sia finita prima dell'esaurimento della passata, che sia stato eseguito o annotato il pagamento della posta; dopodiché qualsiasi possibilità di contestare il risultato è prescritta.
\end{art}
\begin{art}
    Se un giocatore mischia le carte prima che la passata sia ufficialmente conclusa, è obbligatorio tentare di sanare l'errore cercando di ricostruire in maniera sufficiente l'attribuzione dei punti ed anche, quando abbia importanza, la esatta successione temporale degli stessi; se la ricostruzione non è possibile, l'errore non è sanabile ed il responsabile dello stesso, o la parte cui appartiene, risponde dell'errore, anche se commesso in buona fede, riconoscendo all'avversario oltre ai punti accertati anche quelli controversi.
\end{art}
\begin{art}\label{err.pass}
    Se l'errore di cui al precedente paragrafo non è sanabile ed è commesso da giocatori di entrambe le parti, la passata è annullata.
\end{art}
\subsection{Le dichiarazioni}
\begin{art}\label{diritto.dich}
    Il diritto di eseguire una o più dichiarazioni espresse spetta solo al giocatore di mano, al quale non è però consentito di eseguire o completare alcuna dichiarazione dopo che abbia eseguito la giocata di battuta.
\end{art}
\subsubsection{Dichiarazioni relative al palo di pigliata}
\begin{art}
    Con riferimento al palo di pigliata, le dichiarazioni consentite possono essere:
    \begin{packeditem}
\item      \emph{(ho fatto) questo piombo}, quando la presa sia stata effettuata con l'unica o con l'ultima carta del palo in suo possesso; 
\item      \emph{(sono) liscio qui}, quando ancora si possegga almeno una carta soggetta rispetto alle carte possedute al palo dagli altri giocatori.
    \end{packeditem}
\end{art}
\subsubsection{Dichiarazioni relative a pali diversi da quello di pigliata e
di giocata}
\begin{art}
    Con riferimento ai pali diversi da quello di pigliata e da quello al quale ci si accinge a giocare, le dichiarazioni consentite sono:
    \begin{packeditem}
\item      \emph{feci un piombo}, per comunicare di aver esaurito le carte ad uno o più pali diversi da quelli ai quali si sia giocato prima della battuta nella quale il giocatore ha rilevato la mano.
\item      \emph{(sono) faglio a un palo}, per comunicare che, in origine o (indistintamente) al momento della dichiarazione, si era o si è sprovvisti di carte ad uno o più pali.
\item      \emph{faglio al (palo del) mio compagno}, per comunicare di non possedere alcuna carta al palo dove si ritiene che il compagno, attraverso accuse, sfide, o scarti dello stesso o per altre circostanze di gioco, abbia interesse; 
\item      \emph{liscio al (palo del)  mio compagno}, per comunicare il possesso di almeno una carta non sovrana al palo dove si ritiene che il compagno, attraverso accuse, sfide o scarti dello stesso o per altre circostanze di gioco, abbia interesse.
    \end{packeditem}
\end{art}
\begin{art}
    L'erronea dichiarazione di faglio o di liscio al palo del compagno eseguita in buona fede non è sanzionabile.
\end{art}
\subsubsection{Dichiarazioni relative al palo di giocata}
\begin{art}
    Con riferimento al palo al quale ci si accinge a giocare, è consentito dichiarare:
    \begin{packeditem}
\item      il possesso di una Napoletana;
\item      il possesso di un gruppo di due carte di tressette;
\item      il possesso di una singola carta di tressette;
\item      il possesso del Re;
\item      l'eventuale grado sovrano della carta che si gioca;
\item        la semplice lunghezza del palo.
    \end{packeditem}
\end{art}
\paragraph{La dichiarazione della Napoletana}
\begin{art}
    Il giocatore che intende dichiarare il possesso di una Napoletana al palo di giocata può farlo espressamente, giocando una qualsiasi carta e precisando la lunghezza del palo.
\end{art}
\paragraph{Le sfide}
\begin{art}
    Si definisce sfida una dichiarazione relativa ad un palo con due carte di tressette o ad un palo almeno terzo con una sola carta di tressette.
\end{art}
\begin{art}
    Se si possiede una sola di carta di tressette di un palo almeno terzo, la sfida può essere lanciata giocando qualsiasi carta semplice.
\end{art}
\begin{art}
    Se si posseggono due carte di tressette (che vanno entrambe dichiarate):
    \begin{packeditem}
\item se il palo è secondo, la dichiarazione comporta l'obbligo di giocare quella superiore, salvo che il palo sia costituito dal Venticinque, nel qual caso a tale obbligo si sostituisce quello di dover giocare le due carte l'una di seguito all'altra;
\item se il palo dichiarato è terzo, resta solo esclusa la possibilità di giocare una carta semplice;
\item  se il palo è più che terzo, si può eseguire la dichiarazione giocando qualsiasi carta. 
    \end{packeditem}
\end{art}
\paragraph{La dichiarazione del Re}
\begin{art}
    La dichiarazione relativa al possesso del Re, senza giocarlo, è consentita solo giocando una carta semplice di un palo almeno terzo senza carte di tressette.
\end{art}
\paragraph{La dichiarazione di sovranità della carta giocata}
\begin{art}
    La dichiarazione di sovranità della carta giocata può riguardare sia una carta diventata sovrana in assoluto sia una carta diventata tale perché quella o quelle effettivamente sovrane sono possedute anch'esse dal giocatore e non sono né carta di tressette né il Re.
\end{art}
\begin{art}
    L'erronea dichiarazione di sovranità eseguita in buona fede non è sanzionabile salva l'ipotesi che la sovranità della carta dipenda dal possesso di una carta di tressette o del Re.
\end{art}
\begin{art}
    Chi esegue la dichiarazione di sovranità deve sempre precisare la lunghezza del palo.
\end{art}
\paragraph{La semplice dichiarazione di lunghezza del palo}
\begin{art}
    La semplice dichiarazione di lunghezza del palo non può essere eseguita se tra le carte non giocate ve ne sia una di tressette o il Re.
\end{art}
\begin{art}
    Se si dichiara un palo secondo è obbligatorio giocare la carta di grado superiore, anche se le due carte sono consecutive di grado.
\end{art}
\begin{art}
    Se si vuol rendere noto lo stato di unica o di ultima carta del palo a cui si gioca,  è sufficiente dichiarare \emph{sola}, sia che si tratti di carta soggetta, sia che si tratti di carta sovrana; nel secondo caso, però, è possibile dichiarare \emph{piombo}.
\end{art}
\subsubsection{Altre norme concernenti le dichiarazioni relative al palo di
giocata}
\begin{art}
    Se una qualsiasi dichiarazione viene eseguita senza precisare la lunghezza del palo, si sottintende che il palo sia esattamente terzo; se il palo è quarto o più che quarto, occorre precisare ulteriormente la dichiarazione aggiungendo rispettivamente \emph{lungo} o  \emph{lunghissimo}.
\end{art}
\begin{art}
    Nel lanciare una sfida e, più in generale, nell'eseguire una qualsiasi dichiarazione, occorre usare le espressioni, anche di tipo gestuale, in uso al tavolo.
    Se al tavolo si utilizzano, per la stessa fattispecie dichiarativa, più espressioni alternative, è corretto che ciascun giocatore adoperi sempre la stessa espressione; ciò al fine di non generare il dubbio che a dichiarazioni variamente formulate corrispondano fattispecie secondarie diverse che non possono essere palesate.
\end{art}
\section{Norme generali di comportamento}
\subsection{Comportamento dei giocatori}
\begin{art}
    Il giocatore di tressette è obbligato a mantenere durante tutto il tempo dell'incontro, e in particolare durante le fasi attive di gioco, un atteggiamento sereno ed un comportamento gentile e rispettoso verso compagni e avversari.
\end{art}
\begin{art}
    Nelle fasi attive di gioco il giocatore è altresì obbligato:
    \begin{packeditem}
\item      a rimanere impassibile, contenendo la propria soddisfazione o il proprio disappunto, sia in relazione alle carte possedute, sia in relazione alle giocate eseguite dagli altri giocatori, sia in relazione all'esito di ciascuna giocata;
\item      a tenere le carte in modo che non possano essere lette dagli altri giocatori;
\item      a giocare generalmente con speditezza tale da non rendere il gioco opprimente;
\item      ad eseguire giocate e risposte in maniera sempre pacata ed uniforme;
\item      ad astenersi da qualsiasi tipo di commento fino al termine della passata;
\item       a prestare la dovuta attenzione al fine di non incorrere in errori che, pur se sanabili, incidono sempre negativamente sul regolare andamento del gioco
    \end{packeditem}
\end{art}
\subsection{Comportamento degli spettatori}
\begin{art}
    La presenza al tavolo di spettatori è possibile alle seguenti condizioni:
    \begin{packeditem}
\item      che non sia negata da più della metà dei giocatori:
\item      che il numero degli spettatori non sia superiore a due;
\item      che per ogni spettatore ci sia l'assunzione di responsabilità da parte di uno, e uno solo, dei giocatori.
    \end{packeditem}
\end{art}
\begin{art}
    Lo spettatore, una volta ammesso al tavolo, non può esserbe allontanato senza validi motivi.
\end{art}
\begin{art}
    Lo spettatore, pena il possibile allontanamento dal tavolo, deve osservare i seguenti obblighi:
    \begin{packeditem}
\item      deve restare seduto alla sinistra del giocatore che ne ha assunto la responsabilità (o alla sua destra se questi tiene le carte con la mano destra);
\item      non deve chiedere all'altro giocatore a lui vicino di mostrargli le proprie carte;
\item      deve restare assolutamente in silenzio ed impassibile durante le fasi attive del gioco;
\item      deve astenersi dall'effettuare suggerimenti o commenti, anche se richiestigli;
\item      deve limitarsi, a passata conclusa, ad esprimere, con moderazione e se richiestogli, il suo parere;
\item        non deve denunciare, se non a passata conclusa, eventuali irregolarità successive alle fasi preliminari della passata.
    \end{packeditem}
\end{art}
\begin{art}
    Del comportamento dello spettatore risponde, sia agli effetti regolamentari che a quelli disciplinari, il giocatore che ne ha assunto la responsabilità.
\end{art}
\chapter{Il tressette classico}
\subsection{Numero e posizione dei giocatori}
\begin{art}
Nei giochi di tressette si affrontano quattro giocatori schierati in due
coppie contrapposte, a ciascuno dei quali il cartaro deve distribuire dieci carte frazionate in due gruppi di cinque.
\end{art}
\begin{art}
La posizione iniziale al tavolo di ciascun giocatore rispetto agli altri deve essere stabilita dalla sorte; quella di alcuni di essi può successivamente variare in funzione della modalità seguita per la formazione delle coppie.
\end{art}
\begin{art}
La funzione di cartaro in occasione della prima smazzata dell'incontro compete al giocatore designato dalla sorte o a quello più giovane; nelle smazzate successive tale funzione compete al giocatore seduto a sinistra di quello al quale spetta principiare la prima battuta della passata.
\end{art}
\subsection{Modalità di formazione delle coppie}
\begin{art}
Le diverse modalità di formazione delle coppie e di posizione dei giocatori
al tavolo danno luogo alle quattro diverse modalità di gioco così di seguito
contrassegnate: A. (tressette a coppie fisse), B.a. (tressette incrociato a coppie casuali note), B.b.1 (tressette a posizioni casuali e coppie casuali note), B.b.2 (tressette a posizioni casuali e coppie casuali ignote).
La formazione delle coppie può essere:
\begin{packeditem}
\item       A. Prestabilita, cioè a coppie fisse, ed in tal caso i giocatori di ciascuna coppia siedono sempre, salvo quanto previsto per alcune formule di torneo, l'uno di fronte all'altro (tressette incrociato).
\item       B. Del tutto casuale, ossia dipendente ad ogni partita dal posizionamento presso i giocatori di determinate carte; infatti la formazione delle coppie avviene associando il giocatore che possiede il 4 denari con il giocatore che è tenuto a dichiarare di possedere il 5 denari; se un giocatore possiede entrambe le suddette carte questo giocatore si associa col giocatore che è tenuto a dichiarare di possedere il 6 denari; se, infine, un giocatore possiede tutte e tre le succitate carte, la distribuzione deve essere ripetuta. Con questa modalità di formazione delle coppie la passata s'intende iniziata nel momento in cui il giocatore dichiara di possedere il 4 denari anche se la battuta iniziale deve essere eseguita solo dopo che il suo compagno si sia dichiarato. La posizione dei giocatori può essere:
\item B.a. Obbligata, nel senso che i due giocatori di una coppia devono giocare necessariamente l'uno di fronte all'altro e, in tal caso, se occorre, il giocatore che possiede il 5 denari (o il 6 denari) deve, prima che la passata abbia inizio, scambiarsi di posto con il giocatore posizionato di fronte al giocatore che possiede il 4 denari (o il 5 denari).
\item  B.b. Originaria, in quanto tutti i giocatori mantengono sempre il proprio posto sicché ciascuno di essi può trovarsi il compagno alla sua destra (sopramano), alla sua sinistra (sottomano) o di fronte; la formazione delle coppie può avvenire:
\item   C. Condizionata, attraverso la Chiamata del Tre, alla scelta di gioco del giocatore di mano.
    \end{packeditem}
\end{art}
\subsection{Tipi di partita}
\begin{art}
I giochi di tressette si distinguono a seconda che la coppia vinca la partita:
\begin{packeditem}
\item      A punti, ossia raggiungendo prima dell'altra un punteggio prefissato (vittoria ordinaria) o conquistando in una passata tutti gli undici punti di mazzo indipendentemente dai punteggi eventualmente già acquisiti (vincita per cappotto). Il punteggio ordinario per la vittoria è fissato a ventuno punti; si può prevedere un punteggio maggiore purché multiplo di sette.
\item     A passate, ossia conseguendo un punteggio superiore a quello dell'altra coppia in un numero prefissato di passate pari, normalmente, a quattro o a un suo multiplo. Salvo che il numero di passate sia dispari e che si giochi senza accusa, è possibile che la partita si concluda in parità; per tale eventualità può prevedersi che il gioco debba continuare fino a quando la situazione di parità non persista.
\item     A passata singola, ossia raggiungendo i sei punti.
    \end{packeditem}
\end{art}
\subsection{Le Poste e il Piatto}
\begin{art}
La posta può essere dei seguenti tipi:
\begin{packeditem}
\item   Posta unica, qualora venga stabilito che essa sarà sempre tale sia che la partita si vinca ordinariamente e sia che si concluda per cappotto.
\item Posta variabile, qualora venga stabilito che la stessa vari in funzione della differenza punti.
\item Posta semplice o doppia, qualora venga stabilito che si paghi semplice (cioè una sola posta) quando la partita è vinta ordinariamente, e si paghi doppia (cioè due poste) quando la partita è vinta per cappotto.
\item Posta con marcio e batuffa, qualora venga stabilito che oltre alla posta ci si giochi anche il piatto; giocando con tale tipo di posta:
\item   se la partita si conclude ordinariamente, i perditori pagano posta semplice (ossia una posta) se hanno sbatuffato (ossia se i loro punti stanno a 11 o più) e pagano posta doppia (ossia due poste) in caso contrario;
\item   se la partita si conclude per cappotto, i vincitori si dividono in parti uguali il piatto ed esigono posta doppia se gli avversari hanno sbatuffato e posta tripla (ossia tre poste) in caso contrario.
    \end{packeditem}
\end{art}
\begin{art}
Il piatto si alimenta con la messa da parte di ciascun giocatore di una posta prima di dar inizio alla prima partita (o alla prima partita successiva ad un cappotto) e di un quarto di posta prima dell'inizio di ciascuna partita successiva. Se all'incontro partecipano più di quattro giocatori il giocatore o i giocatori che riposavano in occasione della partita conclusasi per cappotto hanno diritto a ritirare quanto dagli stessi versato nel piatto.
\end{art}
\section{Gioco Ordinario: Norme tradizionali}
\subsection{Norme che regolano le attività preliminari e norme che regolano le giocate}
\begin{art}
Si richiamano ed integralmente si confermano le disposizioni degli articoli
da~\ref{passata.iniz} a~\ref{err.pass} del presente Codice, fatta esclusione per quelle norme eventualmente in contrasto con quanto ulteriormente o diversamente sancito nei seguenti articoli.
\end{art}
\begin{art}
Per battuta s'intende il complesso di quattro carte delle quali la prima rappresenta la giocata di battuta e le altre tre le giocate di risposta; la battuta inizia ufficialmente con la giocata o con qualsiasi altra dichiarazione eseguita dal giocatore di mano e cessa con il deposito della pigliata.
\end{art}
\begin{art}
La pigliata deve essere sempre raccolta dal giocatore che la conquista e dallo stesso ordinatamente depositata:
\begin{packeditem}
\item innanzi a sé, se trattasi della prima pigliata conquistata, nella passata, da lui o dal suo compagno;
\item sulle precedenti pigliate della coppia, per le pigliate successive alla prima.
    \end{packeditem}
\end{art}
\begin{art}
Se la formazione delle coppie avviene con il sistema della chiamata del Tre, ciascun giocatore deve depositare presso di sé le pigliate conquistate e non può unirle con quelle del compagno, o presunto tale, fino a quando le coppie non si siano ufficialmente formate, il che avviene solo a seguito di un'accusa o di una sfida o della giocata della carta chiamata e mai per semplice, anche se evidente, deduzione da altri accadimenti di gioco. Formatesi ufficialmente le coppie, le carte depositate innanzi a ciascun giocatore debbono essere riunite con quelle del compagno: tale compito, da eseguirsi a battuta conclusa,   spetta a quello che ha effettuato la sua prima pigliata dopo la prima pigliata del compagno e deve essere eseguito ponendo le proprie carte, senza mutarne l'ordine, su quelle del compagno.
\end{art}
\begin{art}\label{rivedere}
Al giocatore di mano è consentito, prima d'iniziare la battuta, di rivedere le carte depositate innanzi a sé; nell'eseguire tale operazione non deve alterarne l'ordine, né mostrarle al compagno, né eseguire commenti. A coppie ufficialmente formate, gli è anche consentito, purché la successiva battuta rientri nelle prime quattro della passata, di farsi mostrare dagli avversari che la avessero conseguita la prima pigliata in assoluto della stessa.
\end{art}
\subsection{Norme che regolano la messa a terra}
\begin{art}\label{messa.terra}
La messa a terra è l'azione con la quale il giocatore che detiene la mano, o che è chiamato a rispondere su di una giocata iniziata da un altro giocatore, cala sul tavolo tutte le proprie carte rivendicandone la sovranità.
\end{art}
\begin{art}\label{terra1}
Non può essere eseguita dal giocatore in attesa di rilevare la mano mediante la giocata del proprio compagno, quand'anche risultasse assolutamente irrilevante la carta giocata da quest'ultimo perché tutte gli consentirebbero in qualsiasi momento di prendere la mano.
\end{art}
\begin{art}\label{terra2}
La messa a terra è valida ed efficace solo al verificarsi delle seguenti condizioni:
\begin{packeditem}
\item      l'azione, ai sensi di quanto sancito ai punti precedenti, non deve essere stata intempestiva; resta però valida se, eseguita a battuta iniziata ed erroneamente in anticipo rispetto al compagno, si verifichi irrilevante la carta di risposta di quest'ultimo;
\item      tra le carte messe a terra non ve ne deve essere alcuna che possa rimanere soggetta, neanche rispetto al compagno; in altre parole, tutte le carte,  giocate in ordine decrescente devono consentire allo stesso (e non anche al compagno), qualsiasi sia l'ordine in cui possano essere giocate le carte di risposta dagli altri tre giocatori, di effettuare tutte le pigliate rimanenti;
\item       la condizione sub b. deve resistere alla prova contraria anche nell'ipotesi che le (sole) carte incognite degli altri tre giocatori fossero distribuite tra gli stessi in maniera diversa da quella effettiva.
    \end{packeditem}
\end{art}
\begin{art}
In caso di messa a terra intempestiva (condizione sub a.) gli avversari hanno il diritto di annullare la passata.
\end{art}
\begin{art}
Se la messa a terra è risultata non valida per il mancato verificarsi della condizione sub b. o della condizione sub c., il giocatore è obbligato a ritirare le carte e, tenendole scoperte innanzi a sé, a rigiocarle - come si dice - a meglio a meglio, ossia l'una dopo l'altra con ordine decrescente, ma con libertà di scelta tra due carte dello stesso grado; tale obbligo cessa nel momento in cui la mano sia rilevata dal compagno o dagli avversari.
Inoltre, se nel corso delle battute successive la mano viene presa dal compagno di chi aveva effettuato la messa a terra per errore, anche questi, finché resta di mano, dovrà giocare le proprie carte con la sopra indicata modalità.
\end{art}
\begin{art}
Il fatto che il giocatore obbligato a giocare a meglio a meglio realizzi ugualmente tutte le prese non sana l'irregolarità una volta che questa sia stata accertata e la messa a terra, quindi, rimane inefficace ai fini di quanto previsto al punto seguente.
\end{art}
\begin{art}\label{fine.terra}
La messa a terra va considerata come una sola battuta sovrana; tale modalità di conseguimento delle rimanenti pigliate è rilevante:
\begin{packeditem}
\item     al fine di evitare che gli avversari, accusando, possano vincere la partita prima che si esaurisca la prima battuta;
\item    nelle ipotesi di cappotto, al fine di evitare, come previsto
    dall'art.~\ref{chiamata.fuori} di essere chiamati fuori.
    \end{packeditem}
\end{art}
\subsection{Norme che regolano le dichiarazioni}
\begin{art}\label{classic.dich}
S'intende qui confermato e richiamato l'intero contenuto degli articoli
da~\ref{diritto.dich} a~\ref{fine.dich} del presente Codice, fatta esclusione per quelle norme eventualmente in contrasto con quanto ulteriormente o diversamente sancito negli articoli che seguono.
\end{art}
\begin{art}
Il giocatore che inizia tacitamente una battuta con una carta (semplice o di
Tressette) piombo, sola, lisciante, terza, lunga o lunghissima non può
palesare neppure parzialmente lo stato del palo dopo che abbia eventualmente
conquistata la pigliata, e cioè non può dire, accingendosi a giocare ad
altro palo: \emph{ho fatto questo piombo} oppure \emph{sono liscio qui}.
\end{art}
\begin{art}\label{classic.dich2}
Può però, avendo conseguito la pigliata su una sua giocata tacita, rigiocare
--- anche di seguito --- allo stesso palo ancora tacitamente o anche
utilizzando una delle formule di cui al~\ref{classic.dich.list}.
\end{art}
\begin{art}
E potrà anche (riprendendo successivamente la mano su una giocata da altri
iniziata, anche ad un diverso palo) regolarmente riferirsi al palo al quale
aveva giocato in silenzio utilizzando una delle formule di cui
al~\ref{classic.dich.list}.
\end{art}
\begin{art}\label{classic.dich.list}
Le dichiarazioni possibili con riferimento al palo di giocata sono le seguenti:
\begin{packedenum}
\item        \emph{sola}, giocando l’unica o l’ultima carta del palo, purch non
sovrana
\item        \emph{piombo}, giocando l’unica o l’unica carta del palo se trattasi di
carta (anche notoriamente) sovrana
\item        \emph{si liscia}, giocando la carta di tressette (sovrana o soggetta) o,
comunque, la carta di grado superiore (sovrana o soggetta) ad un palo ove è
presente solo un’altra carta semplice
\item        \emph{è mia e si liscia}, giocando la carta sovrana di tressette o,
comunque, la carta sovrana di grado superiore ad un palo ove è presente solo
un’altra carta semplice
\item        \emph{terzo}, giocando una carta sovrana o soggetta diversa dal Re ad un
palo ove siano presenti altre due carte semplici tra cui può esserci o non
esserci il Re
\item       \emph{si liscia due volte}, giocando la carta di tressette (sovrana o
soggetta) o, comunque, la carta di grado superiore (sovrana o soggetta) ad
un palo ove sono  presenti altre due carte semplici
\item        \emph{è mia e si liscia due volte}, giocando la carta sovrana di
tressette o, comunque, la carta sovrana di grado superiore ad un palo ove
sono presenti altre due carte semplici
\item       \emph{è mia, terzo}, giocando una carta di tressette sovrana o una carta
semplice sovrana o soggetta esclusivamente ad una o ad entrambe le altre due
carte semplici possedute dal giocatore ad un palo terzo 
\item        \emph{vero doppio liscio} o \emph{terzo di Re}, giocando una qualsiasi carta
semplice di un palo terzo senza carte di tressette con il Re che non viene
giocato
\item      \emph{lungo}, giocando una carta sovrana o soggetta diversa dal Re ad un
palo ove siano presenti altre tre carte semplici tra cui può esserci o non
esserci il Re
\item     \emph{si liscia, lungo}, giocando la carta di tressette (sovrana o
soggetta) o, comunque, la carta di grado superiore (sovrana o soggetta) ad
un palo ove sono  presenti altre tre carte semplici
\item      \emph{è mia, lungo}, giocando una carta di tressette sovrana o una carta
semplice sovrana o soggetta esclusivamente ad una o a più carte semplici
possedute dal giocatore ad un palo quarto 
\item      \emph{lungo tutta la Napoletana} o \emph{lungo di Re}, giocando una qualsiasi
carta semplice di un palo quarto senza carte di tressette con il Re che non
viene giocato
\item      \emph{lunghissimo}, giocando una carta sovrana o soggetta diversa dal Re
ad un palo ove siano presenti almeno altre quattro carte semplici tra cui
può esserci o non esserci il Re
\item      \emph{si liscia, lunghissimo}, giocando la carta di tressette (sovrana o
soggetta) o, comunque, la carta di grado superiore (sovrana o soggetta) ad
un palo ove sono  presenti almeno altre quattro carte semplici
\item      \emph{è mia, lunghissimo}, giocando una carta di tressette sovrana o una
carta semplice sovrana o soggetta esclusivamente ad una o a più carte
semplici possedute dal giocatore ad un palo quinto
\item      \emph{lunghissimo tutta la Napoletana} o \emph{lunghissimo di Re}, giocando una
qualsiasi carta semplice di un palo almeno quinto senza carte di tressette
con il Re che non viene giocato
\item      \emph{liscio e busso} o \emph{Asso terzo}, giocando una qualsiasi carta semplice
di un palo terzo con l’Asso
\item      \emph{liscio e busso lungo} o \emph{Asso lungo}, giocando una qualsiasi carta
semplice di un palo quarto con l’Asso
\item      \emph{liscio  busso lunghissimo} o \emph{Asso lunghissimo}, giocando una
qualsiasi carta semplice di un palo almeno quinto con l’Asso
\item      \emph{ribusso} o \emph{Due terzo}, giocando una qualsiasi carta semplice di un
palo terzo con il Due
\item      \emph{ribusso lungo} o \emph{Due lungo}, giocando una qualsiasi carta semplice
di un palo quarto con il Due
\item      \emph{ribusso lunghissimo} o \emph{Due lunghissimo}, giocando una qualsiasi
carta semplice di un palo almeno quinto con il Due
\item      \emph{busso} o \emph{Tre terzo}, giocando una qualsiasi carta semplice di un
palo terzo con il Tre
\item      \emph{busso lungo} o \emph{Tre lungo}, giocando una qualsiasi carta semplice di
un palo quarto con il Tre
\item      \emph{busso lunghissimo} o \emph{Tre lunghissimo}, giocando una qualsiasi carta
semplice di un palo almeno quinto con il Tre
\item      \emph{liscio, la migliore} o \emph{ventotto piombo}, giocando il due di un
ventotto piombo
\item      \emph{doppio liscio, la migliore} o \emph{ventotto terzo}, giocando il Due o
l’Asso di un ventotto terzo
\item      \emph{la migliore, lungo} o \emph{ventotto lungo}, giocando una qualsiasi carta
di un ventotto lungo
\item      \emph{la migliore, lunghissimo} o \emph{ventotto lunghissimo}, giocando una
qualsiasi carta di un ventotto lunghissimo
\item      \emph{liscio, la buona} o \emph{ventinove piombo}, giocando il Tre di un
ventinove piombo
\item      \emph{doppio liscio, la buona} o \emph{ventinove terzo}, giocando il Tre o
l’Asso di un ventinove terzo
\item      \emph{la buona, lungo} o \emph{ventinove lungo}, giocando una qualsiasi carta
del ventinove lungo
\item      \emph{la buona, lunghissimo} o \emph{ventinove lunghissimo}, giocando una
qualsiasi carta di un ventinove lunghissimo
\item      \emph{è mio, liscio e piombo} o \emph{venticinque piombo}, giocando
consecutivamente il Due ed il Tre di un venticinque piombo
\item      \emph{è mio, terzo} o \emph{venticinque terzo}, giocando il Due di un venticinque
terzo con o senza il Re
\item      \emph{è mio e voglio l’Asso} o \emph{venticinque terzo di Re}, giocando il Due
di un venticinque terzo con il Re
\item      \emph{è mio, lungo} o \emph{venticinque lungo}, giocando una qualsiasi carta di
un venticinque lungo con o senza il Re
\item      \emph{è mio, lungo e voglio l’Asso} o \emph{venticinque lungo di Re}, giocando
una qualsiasi carta diversa dal Re di un venticinque lungo con il Re
\item      \emph{è mio, lunghissimo} o \emph{venticinque lunghissimo}, giocando una
qualsiasi carta di un venticinque lunghissimo con o senza il Re
\item      \emph{è mio, lunghissimo e voglio l’Asso} o \emph{venticinque
    lunghissimo di Re}, giocando una qualsiasi carta diversa dal Re di un venticinque lunghissimo
con il Re.
\end{packedenum}
\end{art}
\subsection{Il concetto di conformità delle dichiarazioni}
\begin{art}
Si considerano conformi le sole dichiarazioni che, oltre ad essere sostanzialmente corrette, siano anche effettuate in modo chiaro ed utilizzando le formule espressamente previste; a fronte di dichiarazioni conformi non è consentito a nessun giocatore d'interrompere il gioco per chiedere ulteriori precisazioni.
\end{art}
\begin{art}
In caso di dichiarazione non conforme, e anche quando la stessa sia
effettuata in forma non verbale, per quanto ammessa, ciascun giocatore ha il
diritto, prima di giocare la sua carta di risposta, di chiedere che il
giocatore precisi la sua dichiarazione e che, eventualmente, la corregga. Se
a seguito della correzione la dichiarazione risulta sostanzialmente errata,
si applica quanto previsto al~\ref{emersione.anomalie}.
\end{art}
\subsection{Norme che regolano la chiamata fuori}
\begin{art}\label{chiamata.fuori}
Quando un giocatore ritiene che la coppia avversaria, con le sole pigliate, o con le accuse prima e le pigliate poi, abbia raggiunto o superato i 21 punti, o che la stessa, con le sole accuse, abbia raggiunto i ventuno punti senza tuttavia superare i 22, ha il diritto di chiamarla fuori e, quindi, di chiedere la cessazione della partita.
\end{art}
\begin{art}\label{chiamata.fuori2}
La chiamata fuori è lecita anche se chi la effettua, o il suo compagno, abbia nel frattempo rilevato la mano senza per altro aver conseguito il punto.
\end{art}
\begin{art}
La coppia chiamata fuori ha il diritto di veder completata la battuta e, poi, quando non sia già del tutto evidente, di verificare il punteggio raggiunto, comprensivo dei punti conseguiti con l'ultima battuta.
\end{art}
\begin{art}
La verifica del punteggio compete al giocatore che è di mano o, se si gioca escludendo l'ordinaria possibilità di rivedere le pigliate, al suo compagno.
\end{art}
\begin{art}
Se la chiamata fuori proviene da chi abbia rilevato la mano senza aver fatto il punto, o dal compagno di questi, la verifica del punteggio compete sempre all'avversario che segue di giro il giocatore di mano.
\end{art}
\begin{art}
Il giocatore che esegue la verifica è tenuto ad effettuarla scorrendo le carte senza eccessivi indugi, senza mutarne l'ordine e  senza mostrarle agli altri giocatori e, se riscontra di non aver raggiunto i ventuno punti, deve dichiarare esattamente quanti punti e/o quante frazioni di punto mancano al raggiungimento.
\end{art}
\begin{art}
La coppia che, nella stessa passata, effettui una seconda chiamata fuori e dalla verifica risulti che i 21 punti non sono ancora stati raggiunti, perde il diritto ad una terza chiamata fuori per cui gli avversari potranno proseguire il gioco senza dover dichiarare l'andata per cappotto.
\end{art}
\begin{art}
La coppia che consegue il cappotto dopo aver già vinto la partita ordinaria, anche se gli avversari non la abbiano chiamata fuori, vince il piatto ma non ha più il diritto di incassare la posta tripla, bensì la posta semplice o la posta doppia a seconda che gli avversari abbiano o meno sbatuffato.
\end{art}
\begin{art}
La coppia che consegue il cappotto con la messa a terra non può essere chiamata fuori ed avrà quindi il diritto di esigere  posta doppia o  posta tripla a seconda che gli avversari abbiano o meno sbatuffato prima della messa a terra.
\end{art}
\subsection{Norme che regolano la dichiarazione di andata per cappotto}
\begin{art}\label{cappotto1}
La coppia vincente per chiamata fuori ha la facoltà di dichiarare di andare per cappotto; tale dichiarazione può scaturire da una decisione di uno dei due giocatori, unilaterale ma vincolante, oppure da una decisione condivisa da entrambi.
\end{art}
\begin{art}
Alla decisione finale di andare o non andare per cappotto si deve pervenire
attraverso dichiarazioni verbali, immediate ed esplicite, da rilasciarsi
secondo l'ordine temporale indicato al successivo articolo ed utilizzando
esclusivamente le formule verbali indicate all'art.~\ref{cappotto2}. 
\end{art}
\begin{art}\label{cappotto}
Il primo a dover rilasciare la dichiarazione verbale è il giocatore di mano
o il giocatore che abbia raggiunto i 21 punti (senza superare i 22) avendo
accusato al suo turno di risposta in una battuta conquistata dagli avversari
senza che questi abbiano conseguito il punto oppure, nell'ipotesi di cui
all'art.~\ref{chiamata.fuori2}, il giocatore che segue di giro l'avversario di mano.
\end{art}
\begin{art}\label{cappotto2}
Alla dichiarazione finale (di andare o non andare per cappotto) si può pervenire in otto modi diversi:
\begin{packeditem}
\item    il primo giocatore dice non vado (o scopre le carte sul tavolo) = unilaterale negativa;
\item    il primo giocatore dice vado = unilaterale positiva;
\item    il primo giocatore dice non potrei (interlocutoria negativa) ed il compagno dice non andiamo  = negativa condivisa;
\item    il primo giocatore dice non potrei  (interlocutoria negativa) ed il compagno dice vado = unilaterale positiva;
\item    il primo giocatore dice potrei (interlocutoria positiva) ed il compagno dice andiamo = positiva condivisa;
\item    il primo giocatore dice potrei (interlocutoria positiva) ed il compagno dice non vado (o scopre le carte sul tavolo) = unilaterale negativa;
\item    il primo giocatore dice potrei (interlocutoria positiva); il compagno dice non potrei  (interlocutoria negativa); il primo giocatore dice vado = unilaterale positiva;
\item    il primo giocatore dice potrei (interlocutoria positiva); il compagno dice non potrei  (interlocutoria negativa); il primo giocatore dice non andiamo (o scopre le carte sul tavolo) = negativa condivisa.
    \end{packeditem}
\end{art}
\begin{art}
Se si gioca a coppie fisse o se si conviene espressamente di escludere le dichiarazioni unilaterali, alla dichiarazione finale  (di andare o non andare per cappotto) si può pervenire solo nei seguenti tre modi:
\begin{packeditem}
\item    il primo giocatore dice non vado (o scopre le carte sul tavolo);
\item    il primo giocatore dice potrei ed il compagno dice no  (o scopre le carte sul tavolo);
\item    il primo giocatore potrei ed il compagno dice andiamo.
    \end{packeditem}
\end{art}
\begin{art}
A nessuno dei giocatori che devono decidere l'andata per cappotto è consentito di rivedere le proprie pigliate dopo che i punti che gli hanno fatto vincere la partita siano già stati contati o già indubbiamente accertati.
\end{art}
\begin{art}
La facoltà di dichiarare l'andata per cappotto si perde irrimediabilmente
nel caso la verifica del punteggio sia stata eseguita dal giocatore cui non
competeva e nei casi di inosservanza di quanto sancito dagli articoli
da~\ref{cappotto1}
a~\ref{cappotto2};  e si perde, altresì, se le dichiarazioni non sono effettuate con tono assolutamente neutro nonché in caso di tentennamenti o di indugi prolungati o di  ricostruzioni ad alta voce delle battute e delle dichiarazioni verbali precedenti, ecc.
\end{art}
\begin{art}
La coppia che non consegue il cappotto dopo averlo dichiarato, esige la posta semplice o doppia vinta prima della dichiarazione, ma raddoppia la somma che si trova nel piatto.
\end{art}
\begin{art}
Quando si gioca a cinque persone, i vincitori del cappotto incassano ciascuno due quinti della somma esistente nel piatto, e rilasciano un quinto a chi è in turno di riposo; se i giocatori sono sei, i vincitori prendono ognuno i due sesti e rilasciano un sesto a ciascuno dei due giocatori che riposano. Viceversa, la coppia che abbia dichiarato il cappotto senza poi conseguirlo, raddoppia la somma nel piatto, decurtata però del quinto se i giocatori sono cinque o del sesto se sono sei.
\end{art}
\subsection{Norme che regolano i risarcimenti}
\begin{art}
Nel Tressette Ordinario i giocatori possono di comune accordo prevedere che, se uno di essi determina colpevolmente il verificarsi per la coppia di un risultato negativo, tale giocatore debba risarcire il compagno del danno causatogli.
\end{art}
\begin{art}
Il risarcimento è circoscritto alla posta o alle poste che il compagno è stato costretto ad esitare alla coppia avversaria e per la ricostituzione o il raddoppio del piatto, e non si estende alla mancata vincita pur se la stessa poteva essere sicuramente realizzata.
\end{art}
\begin{art}
Perché si configuri l'obbligo del risarcimento occorre la coesistenza dell'evento negativo e del comportamento colpevole; sia l'uno che l'altro sono esclusivamente quelli previsti nei due successivi due articoli.
\end{art}
\begin{art}
Ai fini del determinarsi dell'obbligo del risarcimento, si considerano eventi negativi:
\item  il cappotto subito;
\item  il cappotto dichiarato e non realizzato.
\end{art}
\begin{art}
Ai fini del determinarsi dell'obbligo del risarcimento un comportamento è considerato colpevole quando il giocatore:
\begin{packeditem}
\item  essendo di mano, e possessore di un Venticinque o di una Napoletana, anziché giocarne subito un pezzo, ed anche due, per fare il punto, dia corso ad una giocata diversa;
\item  essendo di mano, e possessore di un solo Tre e di qualche carta secondaria di Tressette, lanci la sfida nel palo dell'unico Tre;
\item  essendo di mano, e possessore di due Tre, o di un Tre ed un Ventinove, lanci una sfida di Tre o una sfida di Ventinove giocando una carta semplice;
\item  rifiuta la pigliata, con il punto visibile a terra o anche senza punto ma con la possibilità di battere una carta sovrana o di portare il gioco ad una carta sovrana del compagno resa nota col palesamento del buongioco o di una sfida;
\item  avendo conquistato una pigliata senza punto, non giochi l'unico Tre o altra carta sovrana che possiede;
\item  conquista la pigliata con l'unico Tre (non piombo) che possiede senza realizzare il punto;
\item  essendo possessore di due Tre ed avendo conquistato con uno di essi una pigliata senza punto, non giochi il secondo Tre;
\item  essendo possessore di tre Tre, ed avendone giocato uno sulla bussata di un avversario, senza aver fatto il punto, bussi nel palo di uno degli altri due Tre rimastigli;
\item  pur potendo, non rilasci una figura o un Asso sulla pigliata conquistata o iniziata dal compagno con una carta sovrana e, perciò, non fa conseguire il punto;
\item              avendo lui o il compagno dichiarato di andare per cappotto, ed avendo le carte occorrenti per conseguirlo, metta erroneamente a terra ai sensi di quanto previsto dagli articoli~\ref{terra1} e~\ref{terra2};
\item             non consegue il cappotto deciso unilateralmente in uno dei
    modi previsti dall'art.~\ref{cappotto2} ai punti 2, 4 e 7.
    \end{packeditem}
\end{art}
\section{Gioco Ordinario: norme moderne}
\begin{art}
La normativa tradizionale può essere opportunamente modificata adottando una o più norme moderne.
La normativa moderna non può essere integrata o modificata in alcun modo.
\end{art}
\subsection{Norme che regolano le attività preliminari}
\begin{art}
Le carte vanno rimischiate e la distribuzione ripetuta:
\begin{packeditem}
\item    qualora i giocatori non siano tutti presenti, poiché gli assenti verrebbero ad essere privati della facoltà di mischiare anch'essi le carte prima della distribuzione;
\item    qualora durante la distribuzione si scopra anche una sola carta semplice.
\item    qualora ne cada qualcuna sul pavimento, anche quando non si vede quale sia;
\item    qualora uno dei giocatori riceva un numero di carte diverso da quello previsto o, pur nel numero esatto, le riceva con modalità diverse da quelle previste.
\item      quando, a gioco in corso, venga riscontrato che un giocatore aveva ricevuto un numero di carte inferiore o superiore a quello previsto.
    \end{packeditem}
\end{art}
\subsection{Norme che regolano le giocate}
\begin{art}
Circa la possibilità di rivedere le pigliate precedenti è esclusa la
previsione di cui all'art.~\ref{rivedere}.
\end{art}
\subsection{Norme che regolano la messa a terra}
\begin{art}
Si richiamano e integralmente si confermano le disposizioni degli articoli
da~\ref{messa.terra} a~\ref{fine.terra} del presente Codice.
\end{art}
\subsection{Norme che regolano le dichiarazioni}
\begin{art}
Si richiamano e integralmente si confermano le disposizioni degli articoli
da~\ref{diritto.dich} a~\ref{fine.dich} e da~\ref{classic.dich}
a~\ref{classic.dich.list} del presente Codice, fatta esclusione per quelle norme eventualmente in contrasto con quanto ulteriormente o diversamente sancito nei seguenti articoli.
\end{art}
\begin{art}
L'eventuale dichiarazione relativa al palo di pigliata può essere eseguita solo al momento in cui la pigliata è raccolta e depositata e mai dopo che sia stata eseguita un'altra dichiarazione relativa ad un altro palo o che sia stata iniziata la battuta successiva.
\end{art}
\begin{art}
Il giocatore che abbia dichiarato di essere liscio al palo di pigliata è obbligato a principiare la battuta successiva giocando ad un altro palo.
\end{art}
\begin{art}
L'eventuale dichiarazione relativa ad un palo diverso da quello di pigliata e da quello di giocata può essere eseguita solo prima dell'eventuale dichiarazione relativa al palo di giocata o della giocata eseguita senza alcuna dichiarazione.
\end{art}
\begin{art}
Tutte le dichiarazioni di \emph{faglio} e la dichiarazione \emph{feci un piombo} non sono consentite se riferite ad un palo al quale lo stato di piombo del giocatore è notorio in virtù delle sue giocate o, anche, in virtù delle carte uscite e/o delle dichiarazioni al palo eseguite da tutti i giocatori.
\end{art}
\begin{art}
La dichiarazione di \emph{liscio al mio compagno} non è consentita se il possesso di almeno una carta che giustificherebbe tale dichiarazione è notorio in virtù delle carte uscite e/o delle dichiarazioni al palo eseguite da tutti i giocatori.
\end{art}
\begin{art}
L'eventuale dichiarazione relativa al palo di giocata, che preclude l'esecuzione di ulteriori dichiarazioni, può essere eseguita solo prima della giocata o contemporaneamente ad essa.
\end{art}
\begin{art}
    A differenza di quanto previsto dall'art.~\ref{classic.dich2}, il giocatore che abbia principiato la battuta in silenzio, ossia senza averla accompagnata con la dichiarazione relativa al palo di giocata, non può più, finché è di mano, eseguire nessuna dichiarazione relative a tale palo.
\end{art}
\begin{art}
Il giocatore che abbia effettuato una dichiarazione relativa al palo di giocata non può più nel corso dell'intera passata rilasciare dichiarazioni che si riferiscano allo stesso palo, fatta salva l'ipotesi contemplata nell'articolo seguente.
\end{art}
\begin{art}
Il giocatore che abbia effettuato una qualsiasi dichiarazione di lunghissimo può sempre comunicare l'eventuale stato ancora lunghissimo dello stesso palo.
\end{art}
\begin{art}
La dichiarazione di sovranità della carta giocata non è consentita se lo stato di sovranità è notorio.
\end{art}
\begin{art}
Le dichiarazioni possibili con riferimento al palo di giocata  sono le seguenti:
\begin{packedenum}
\item \emph{sola},  giocando l'unica o l'ultima carta del palo, purché non sovrana
\item \emph{piombo},  giocando l'unica o l'unica carta del palo se trattasi di carta (anche notoriamente) sovrana
\item \emph{si liscia},  giocando la carta di tressette (sovrana o soggetta) o, comunque, la carta di grado superiore (sovrana o soggetta) ad un palo ove è presente solo un'altra carta semplice
\item \emph{terzo},  giocando una carta sovrana o soggetta ad un palo ove siano presenti altre due carte semplici con esclusione del Re
\item \emph{vero doppio liscio} o \emph{terzo di Re},  giocando una qualsiasi carta semplice di un palo terzo senza carte di tressette ma con il Re che non viene giocato
\item \emph{è mia, terzo},  giocando una carta semplice soggetta esclusivamente ad una o ad entrambe le altre carte semplici possedute dal giocatore ad un palo terzo
\item \emph{lungo},  giocando l'unica carta di tressette che si possiede ad un palo quarto, o il Re di un palo quarto senza carte di tressette, o una qualsiasi carta semplice di un palo quarto senza carte di tressette e senza il Re
\item \emph{lungo, tutta la Napoletana} o \emph{lungo di Re},  giocando una qualsiasi carta semplice di un palo quarto senza carte di tressette con il Re che non viene giocato
\item \emph{è mia, lungo},  giocando una carta semplice soggetta esclusivamente ad una o più delle altre carte semplici possedute dal giocatore ad un palo quarto
\item \emph{lunghissimo},  giocando l'unica carta di tressette che si possiede ad un palo almeno quinto, o il Re di un palo almeno quinto senza carte di tressette, o una qualsiasi carta semplice di un palo almeno quinto senza carte di tressette e senza il Re
\item \emph{lunghissimo tutta la Napoletana} o \emph{lunghissimo di Re},  giocando una qualsiasi carta semplice di un palo almeno quinto senza carte di tressette con il Re che non viene giocato
\item \emph{è mia, lunghissimo},  giocando una carta semplice soggetta esclusivamente ad una o più delle altre carte semplici possedute dal giocatore ad un palo almeno quinto
\item \emph{liscio e busso} o \emph{Asso terzo},  giocando una qualsiasi carta semplice di un palo terzo con l'Asso
\item \emph{liscio e busso lungo} o \emph{Asso lungo},  giocando una qualsiasi carta semplice di un palo quarto con l'Asso
\item \emph{liscio  busso lunghissimo} o \emph{Asso lunghissimo},  giocando una qualsiasi carta semplice di un palo almeno quinto con l'Asso
\item \emph{ribusso} o \emph{Due terzo},  giocando una qualsiasi carta semplice di un palo terzo con il Due
\item \emph{ribusso lungo} o \emph{Due lungo},  giocando una qualsiasi carta semplice di un palo quarto con il Due
\item \emph{ribusso lunghissimo} o \emph{Due lunghissimo},  giocando una qualsiasi carta semplice di un palo almeno quinto con il Due
\item \emph{busso} o \emph{Tre terzo},  giocando una qualsiasi carta semplice di un palo terzo con il Tre
\item \emph{busso lungo} o \emph{Tre lungo},  giocando una qualsiasi carta semplice di un palo quarto con il Tre
\item \emph{busso lunghissimo} o \emph{Tre lunghissimo},  giocando una qualsiasi carta semplice di un palo almeno quinto con il Tre
\item \emph{liscio, la migliore} o \emph{ventotto piombo},  giocando il due di un ventotto piombo
\item \emph{doppio liscio, la migliore} o \emph{ventotto terzo},  giocando il Due o l'Asso di un ventotto terzo
\item \emph{la migliore, lungo} o \emph{ventotto lungo},  giocando una qualsiasi carta di un ventotto lungo
\item \emph{la migliore, lunghissimo} o \emph{ventotto lunghissimo},  giocando una qualsiasi carta di un ventotto lunghissimo
\item \emph{liscio, la buona} o \emph{ventinove piombo},  giocando il Tre di un ventinove piombo
\item \emph{doppio liscio, la buona} o \emph{ventinove terzo},  giocando il Tre o l'Asso di un ventinove terzo
\item \emph{la buona, lungo} o \emph{ventinove lungo},  giocando una qualsiasi carta del ventinove lungo
\item \emph{la buona, lunghissimo} o \emph{ventinove lunghissimo},  giocando una qualsiasi carta di un ventinove lunghissimo
\item \emph{è mio, liscio e piombo} o \emph{venticinque piombo},  giocando consecutivamente il Due ed il Tre (nell'ordine) di un venticinque piombo
\item \emph{è mio, terzo} o \emph{terzo, voglio l'Asso},  giocando il Due di un venticinque terzo con o senza il Re
\item \emph{è mio, lungo} o \emph{lungo, voglio l'Asso},  giocando una qualsiasi carta di un venticinque lungo con o senza il Re
\item \emph{è mio, lunghissimo} o \emph{lunghissimo, voglio l'Asso},  giocando una qualsiasi carta di un venticinque lunghissimo con o senza il Re.
\end{packedenum}
\end{art}
\subsection{Norme che regolano la chiamata fuori}
\begin{art}
La coppia del giocatore che avesse esercitato ai sensi
dell'art.~\ref{chiamata.fuori} il diritto di chiamare fuori gli avversari, perde comunque la partita anche se la coppia avversaria, non avendo ancora raggiunto il punteggio previsto per la vittoria, non lo raggiungesse nemmeno nel prosieguo della passata.
La coppia chiamante sarà pertanto tenuta a pagare posta semplice o doppia a seconda del punteggio da essa realizzato fino a quel momento, salvo a dover corrispondere un'ulteriore posta se la coppia chiamata fuori realizzasse il cappotto con una messa a terra eseguita prima di aver raggiunto il punteggio previsto per la vittoria ordinaria.
\end{art}
\begin{art}
Nel caso di chiamata fuori rilevatasi intempestiva:
\begin{packeditem}
\item    la coppia chiamante non potrà più chiamare fuori gli avversari ma
\item    la coppia chiamata avrà l'obbligo di chiamarsi fuori non appena esaurita la battuta al termine della quale avesse raggiunto il punteggio previsto per la vittoria e dovrà, quindi, eseguire le  dichiarazioni  previste per l'andata per cappotto; in caso contrario, l'eventuale cappotto sarebbe annullato.
    \end{packeditem}
\end{art}
\subsection{Norme che regolano i risarcimenti}
\begin{art}
Nel tressette moderno sono escluse tutte le ipotesi di risarcimento.
\end{art}
\section{Il Buongioco}
\subsection{Nozione di Buongioco}
\begin{art}
    Ha \emph{buongioco} il giocatore che possiede una o più combinazioni di carte ciascuna delle quali è costituita da una napoletana o da almeno tre carte di tressette dello stesso grado; una carta di tressette può essere presente in entrambe le combinazioni.
\end{art}
\begin{art}
In ordine al buongioco è possibile giocare:
\begin{packeditem}
\item    con l'accusa, se il buongioco può essere accusato, anche parzialmente;
\item    senza accusa, se il buongioco deve essere taciuto;
\item    con dichiarazione consentita, se il buongioco può essere dichiarato (anche parzialmente) ma, comunque, non si intende accusato.
\item    con dichiarazione obbligatoria, se il buongioco deve essere (interamente) dichiarato anche se non può intendersi come accusato.
    \end{packeditem}
\end{art}
\subsection{Il gioco con l'Accusa}
\begin{art}\label{buongioco}
Se si gioca con l'accusa, che è sempre facoltativa, il buongioco attribuisce
alla coppia del giocatore che correttamente lo denunci nei tempi e nei
termini precisati nei successivi articoli i seguenti punti \emph{di buongioco}:
\begin{packeditem}
\item    tre punti, per la denuncia di una napoletana o di tre carte di tressette dello stesso grado;
\item    quattro punti, per la denuncia delle quattro carte di tressette dello stesso grado.
    \end{packeditem}
\end{art}
\begin{art}
Il giocatore che possiede tutte e quattro le carte di tressette dello stesso grado non può accusarne solo tre.
\end{art}
\begin{art}
Se l'accusa riguarda più buongioco, anche se ottenuti utilizzando una stessa carta in due combinazioni, i punteggi di buongioco si cumulano.
\end{art}
\begin{art}
Se le carte rese note attraverso l'accusa di più di due buongioco implicano la presenza di un altro buongioco non dichiarato, questo si intende comunque accusato.
\end{art}
\begin{art}
Il giocatore in possesso di più di un buongioco ha la facoltà di accusarne solo alcuni, e ciò anche se il buongioco taciuto consta di una carta facente parte di un buongioco accusato.
\end{art}
\begin{art}
Il giocatore che accusa un buongioco deve precisare il palo della napoletana o, se il buongioco è costituito da tre carte dello stesso grado, il palo della carta di pari grado delle tre accusate che non possiede; se dimentica di precisare il palo, l'accusa è incompleta; se precisa un palo diverso da quello dovuto l'accusa è errata.
\end{art}
\begin{art}
A prima battuta ancora in corso nessun giocatore, tranne chi ha palesato il buongioco, può rilevare l'incompletezza o denunciare l'erroneità dell'accusa e nemmeno può, quand'anche l'accusa fosse stata chiaramente formulata in maniera corretta, richiederne la conferma.
Se in tale irregolarità incorre uno degli avversari di chi aveva palesato il buongioco, l'accusa del buongioco può essere immediatamente integrata o corretta con sanatoria definitiva dell'errore originario.
Se in essa incorre il compagno di chi aveva palesato il buongioco, gli avversari hanno la facoltà di andare a monte; se i predetti non si avvalgono di tale facoltà, ha luogo una sanatoria esplicita definitiva obbligatoria che comporta il divieto di rettificare l'accusa incompleta o errata e la perdita dei punti di buongioco se l'accusa fosse stata rettificata, nonché il divieto per entrambi i giocatori di effettuare altre possibili accuse oltre a quelle eventualmente già regolarmente effettuate.
\end{art}
\begin{art}
Il giocatore che vuole accusare un buongioco deve farlo, nella precisa
formulazione descritta all'articolo~\ref{buongioco}, quando giunge il suo primo turno di gioco della passata e non dopo che abbia eseguito la sua prima giocata, di battuta o di risposta; in caso contrario l'accusa è da considerarsi intempestiva.
\end{art}
\begin{art}
Il giocatore che al suo turno di gioco dimentica di accusare o accusa in
maniera incompleta o in maniera errata, può --- a prima battuta ancora in
corso --- rimediare all'omissione oppure può completare o correggere l'accusa irregolarmente eseguita, dando così luogo ad una sanatoria implicita definitiva obbligatoria che comporta per i soli avversari il diritto di sostituire la carta eventualmente già giocata purché trattasi di una giocata di risposta. Tale sanatoria non è valida se gli avversari, avendo accusato nel frattempo, hanno già vinto la partita.
\end{art}
\subsection{Anomalie dell'accusa}
\begin{art}
Se un giocatore, prima che sia giunto il suo turno di gioco, dichiara anche
in maniera incompleta di possedere buongioco, e quand'anche effettivamente
non lo possegga, gli avversari hanno la facoltà di andare a monte; se tale
facoltà non viene tempestivamente esercitata da nessuno degli avversari (si
tenga anche presente che una giocata --- di battuta o di risposta --- equivale a rinuncia tacita del giocatore ad avvalersi della facoltà), si verifica una sanatoria implicita definitiva obbligatoria o, se l'irregolarità viene contestata a battuta appena conclusa, una sanatoria esplicita definitiva obbligatoria che, però, comporta, a differenza di quella implicita, le seguenti conseguenze sulla passata:
\begin{packeditem}
\item    per chi ha commesso l'errore: divieto di completare l'accusa intempestiva e perdita dei punti di buongioco se l'accusa era già stata completata, nonché divieto di effettuare altre possibili accuse;
\item    per il suo compagno: divieto di effettuare nuove accuse, di sanare eventuali accuse incomplete già eseguite, di uscire fino al termine della passata al palo della napoletana anticipatamente accusata (se non perché costrettovi dal non possedere nessuna carta agli altri pali);
\item     per gli avversari: diritto di sostituire la carta eventualmente già giocata purché trattasi di una giocata di risposta e non di battuta.
    \end{packeditem}
\end{art}
\begin{art}
Se un giocatore, esauritasi la prima battuta, esegue un'accusa o integra o corregge una sua accusa incompleta o errata, gli avversari hanno la facoltà di andare a monte; se tale facoltà non viene tempestivamente esercitata si verifica una sanatoria implicita definitiva obbligatoria o, se l'irregolarità viene contestata immediatamente dopo la conclusione della battuta in corso, una sanatoria esplicita definitiva obbligatoria che, però, comporta, a differenza di quella implicita, le seguenti conseguenze sulla passata:
\begin{packeditem}
\item    per chi ha commesso l'errore: perdita dei punti di buongioco;
\item    per il suo compagno: divieto di uscire fino al termine della passata al palo della napoletana anticipatamente accusata (se non perché costrettovi dal non possedere nessuna carta agli altri pali);
\item     per gli avversari: diritto di sostituire la carta eventualmente già giocata purché trattasi di una giocata di risposta e non di battuta.
    \end{packeditem}
\end{art}
\begin{art}
Se nel corso della prima battuta, in conseguenza delle giocate effettuate o della eventuale dichiarazione espressa del giocatore di mano o di un'accusa eseguita da un altro giocatore,
\begin{packeditem}
\item    viene a rendersi palese la carta fagliante al giocatore che aveva eseguito in maniera incompleta l'accusa di tre carte di tressette dello stesso grado: ha luogo una sanatoria implicita definitiva obbligatoria che comporta per i soli avversari il diritto di sostituire la carta di risposta eventualmente già giocata;
\item    si constata l'erroneità di una qualsiasi accusa, napoletana esclusa: l'accusa deve essere corretta e, quindi, ha luogo una sanatoria implicita definitiva obbligatoria che comporta per i soli avversari il diritto di sostituire la carta di risposta eventualmente già giocata.
\item     si constata l'erroneità dell'accusa di una napoletana: l'accusa non può essere corretta e il compagno di chi ha eseguito l'accusa errata non può mai uscire, finché non vi sia  costretto, ai pali ai quali non può escludere, in base alle carte possedute o conosciute in possesso degli avversari, che il suo compagno abbia una napoletana. 
    \end{packeditem}
\end{art}
\end{document}
