\documentclass[italian,a4paper]{article}
\usepackage{babel,url,multirow}
\usepackage[text={6in,9in},centering]{geometry}
\usepackage[utf8x]{inputenc}
\usepackage[T1]{fontenc}
\usepackage{ae,aecompl}
\frenchspacing
\pagestyle{plain}
%------------- eliminare prime e ultime linee isolate
\clubpenalty=9999%
\widowpenalty=9999
%------------- ridefinizione simbolo per elenchi puntati: en dash
\renewcommand{\labelenumi}{\arabic{enumi}.}
\renewcommand{\labelitemi}{--}
%------------- nuovi environment senza spazi
\newenvironment{packeditem}{
\begin{itemize}
  \setlength{\itemsep}{1pt}
  \setlength{\parskip}{0pt}
  \setlength{\parsep}{0pt}
}{\end{itemize}}
\newenvironment{packedenum}{
\begin{enumerate}
  \setlength{\itemsep}{1pt}
  \setlength{\parskip}{0pt}
  \setlength{\parsep}{0pt}
}{\end{enumerate}}
\newenvironment{packeddescr}{
\begin{enumerate}
   \setlength{\itemsep}{1pt}
   \setlength{\parskip}{0pt}
   \setlength{\parsep}{0pt}
}{\end{enumerate}}
%---------

\title{La tecnica segnaletica}
\author{\url{www.fitres.it}}
\date{}

\begin{document}
\maketitle
La tecnica del gioco del tressette si compendia nelle dichiarazioni e nelle
giocate.

Le dichiarazioni, giusta quanto espressamente previsto dal codice, non
possono essere personalizzate. Il giocatore, cioè,  può semplicemente
scegliere  tra due alternative: fare o non fare la dichiarazione che in quel
momento gli è consentita dal Codice; nell'affermativa, però, deve farla solo
in uno dei  modi codificati. Le giocate, invece, a parte l'obbligo di rispondere al palo, sono sempre
libere.

Il discorso tecnico, quindi, si risolve, ad ogni giocata, nell'operare una
duplice scelta: dichiarare o meno ciò che gli è consentito e individuare la
carta più opportuna da giocare tra due o più possibilità.

L'una e l'altra scelta dipendono dallo stato delle proprie carte e, ancor
più, dalla presumibile distribuzione delle altre carte tra gli altri tre
giocatori e, in particolare, dalle carte possedute dal compagno.

 \`E fondamentale, quindi, che i componenti della coppia si scambino quante
 più informazioni è possibile, sebbene tali informazioni, non essendo
 segrete, possano essere apprese anche dagli avversari più attenti che,
 però, ne traggono solitamente un minor beneficio.

 Più informazioni la coppia acquisisce, e prima le acquisisce, maggiori
 saranno le possibilità di incrociare il gioco, di rafforzare la difesa
 nelle passate difficili, di ottimizzare il risultato nelle passate più
 ricche di buone carte.

 Ovviamente, ogni giocata del compagno, e degli avversari, è di per s una
 informazione, ma alcune giocate, anzi la maggior parte, devono essere
 considerate veri e propri ``segnali''.

 L'insieme di questi segnali prende appunto il nome di \emph{tecnica segnaletica
 del tressette}, che è l'argomento di questo quaderno.

I compagni si inviano segnali, potendo così scambiarsi molteplici informazioni, utilizzando cinque strumenti convenzionali:
\begin{packeditem}
\item Dichiarazioni espresse (sfide, dichiarazioni di faglio o di piombo o di liscio o di lunghezza del palo);
\item Scarti;
\item Giocate di accettazione o di rifiuto;
\item Controscarti fissi;
\item Napoletana convenzionale
\end{packeditem}
 
Se i primi tre costituiscono i classici segnali da sempre generalmente adoperati, gli ultimi due, che per naturale contrapposizione linguistica sono stati definiti di tipo scientifico, appartengono all'uso dei giocatori più sofisticati.
Il termine scientifico, nato per definire una particolare modalità di esecuzione di alcune giocate, si è poi riflesso, per metonimia, all'intero impianto del gioco quando questo debba svolgersi soddisfacendo ad esigenze di assoluto rigore non solo tecnico ma anche regolamentare.
\section{I segnali classici}
\subsection{Le dichiarazioni espresse}
Le dichiarazioni espresse sono quelle fatte verbalmente o, con analoghi effetti, in modo simbolico.
La possibilità di fare dichiarazioni rappresenta un grosso vantaggio per il giocatore di mano che, infatti, se ne avvale quasi sempre.
L'omissione di una dichiarazione consentita, pertanto, deve essere considerata dal compagno di per sé come un segnale.
Per conoscere quali siano le dichiarazioni ammesse e quando è possibile rilasciarle, si rinvia al Codice del Tressette.
\subsection{Gli scarti}
Esegue uno scarto il giocatore che, dovendo rispondere una o più volte a battute del compagno o degli avversari, non è in possesso di carte al palo di battuta, ossia è faglio a detto palo.
 
Lo scarto, quindi, non è un atto di volontà ma di necessità. Ciò non giustifica, però, l'apatia con cui alcuni giocatori lo eseguono specie quando posseggono carte inefficaci in tutti i pali. Anche la propria debolezza, infatti, deve essere comunicata al compagno con chiarezza.
 
Gli scarti si distinguono in espliciti, che sono indubbiamente i più efficaci, ed impliciti.
\subsubsection{Gli scarti espliciti}
\begin{description}
\item[Scarto diretto:] consiste nel rilasciare, su due o più giocate sovrane del compagno, due o più carte dello stesso palo in ordine ascendente.
Chi effettua tale scarto invita il compagno a giocare al palo così scartato.
\item[Scarto indiretto:] consiste nel rilasciare, su due o più giocate sovrane degli avversari, due o più carte in ordine decrescente.
 
Chi effettua tale scarto intende comunicare al compagno di possedere un 25, un 3 o anche un 2 al palo così scartato.
 
Se entrambi i compagni eseguono lo scarto indiretto allo stesso palo, è opportuno che quello dei due che abbia forza in un altro palo concentri su questo le sue forze abbandonando il palo indicato dal compagno attraverso lo scarto indiretto.
\item[Scarto di 28:] consiste nel rilasciare, su due o più giocate sovrane degli avversari, prima il 2 del 28 e poi una o due carte semplici dello stesso palo.
Chi effettua tale scarto dichiara inequivocabilmente di possedere un 28 almeno lungo di Re, in mancanza del quale deve limitarsi ad eseguire lo scarto indiretto
\item[Scarto di 3:] sia se effettuato su una battuta degli avversari che su una battuta del compagno, il giocatore dichiara inequivocabilmente di possedere la carta sovrana a detto palo.
\end{description}
\subsubsection{Gli scarti impliciti}
\begin{description}
    \item[Scarto di eliminazione:] 
consiste nel rilasciare, sia sul gioco del compagno che su quello degli avversari, prima una carta del palo debolissimo e poi una carta del palo debole in modo da indicare, per esclusione, il suo palo più forte.
    \item[Scarto di faglio:] 
consiste nel rilasciare, sia su battute sovrane del compagno che su battute degli avversari, due carte dello stesso palo e, quindi, una o due carte dell'unico altro palo che si possiede.
 
Questo scarto va eseguito sempre quando non si sia potuto palesare il faglio con la parola, per non essere stato di mano o per non aver avuto l'occasione di rilevarla.
 
Se lo scarto di faglio viene eseguito su battute sovrane
\emph{del compagno}:
\begin{packeditem}
\item e i due pali che possiede sono l'uno debole e l'altro forte, deve rilasciare prima due carte con ordine decrescente del palo debole e poi una carta, o due in ordine crescente, del palo al quale desidera che il compagno giochi;
\item e i due pali che possiede sono entrambi forti, deve rilasciare due carte in ordine crescente al palo ove è meno forte e poi una carta, o due in ordine crescente, all'unico altro palo (il più forte) che possiede;
\item e i due pali che possiede sono entrambi deboli, deve rilasciare due carte in ordine decrescente al palo ove è più debole e poi una carta, o due in ordine decrescente, all'unico altro palo che possiede.
\end{packeditem}
 
Se lo scarto di faglio viene eseguito su battute sovrane
\emph{dell'avversario}:
\begin{packeditem}
\item e i due pali che possiede sono l'uno debole e l'altro forte, deve rilasciare prima due carte con ordine crescente del palo debole e poi una carta, o due in ordine decrescente, del palo forte;
\item e i due pali che possiede sono entrambi forti, deve rilasciare due carte in ordine decrescente al palo ove è meno forte e poi una carta, o due in ordine decrescente, all'unico altro palo (il più forte) che possiede.
\item e i due pali che possiede sono entrambi deboli, deve rilasciare due carte in ordine crescente al palo ove è più debole e poi una carta, o due in ordine crescente, all'unico altro palo che possiede.
\end{packeditem}
 
Se si è fagli a due pali e, quindi, si è costretti a scartare almeno tre volte all'unico palo che si possiede, tale scarto deve essere sempre crescente su battute del sovrane compagno e sempre decrescente su battute sovrane dell'avversario, anche in assenza di carte di tressette.
Tale eccezione è prevista per non creare equivoci con lo scarto di spoglio e faglio e, quindi, solo se si pratica tale tipo di scarto.
    \item[Scarto di spoglio e faglio:] 
è mutuato dalla tecnica dei controscarti, rappresenta un utile perfezionamento dell'ordinario scarto di faglio che, come si è visto, si esegue su almeno tre battute certamente sovrane, del compagno o degli avversari.

Sulle tre battute del compagno, salvo il caso che si sia fagli a due pali, la carta rilasciata sulla terza battuta, ed eventualmente anche quella rilasciata sulla quarta, appartiene proprio al palo più forte.
L'ordinario scarto di faglio, quindi, comporterebbe il sacrificio obbligato di una carta proprio del palo più forte, carta che potrebbe far guadagnare una battuta, con l'impossibilità di liberarsi, invece, di una residua carta (soggetta) al palo dei precedenti due scarti.  
Spesso, tale spreco basta a non far conseguire il cappotto!

La soluzione consiste nello spoglio del palo più debole, ossia nello scartare solo a detto palo. Per far comprendere, poi, qual è, tra due, l'unico palo che si possiede:

\emph{sul compagno}, occorre scartare prima la carta più alta (in modo da escludere lo scarto diretto) e, quindi, le due successive secondo il seguente ordine:
\begin{packeditem}
\item maggiore, minore: per indicare il palo di rango\footnote{Come
vedremo nel capitolo dedicato ai controscarti, l'ordine decrescente di rango
dei quattro pali è: denari, spade, coppe, bastoni.} minore;
\item minore, maggiore: per indicare il palo di rango maggiore.
\end{packeditem}
\emph{sull'avversario}, occorre scartare prima la carta più bassa (in modo da escludere subito lo scarto indiretto) e, quindi, le due successive secondo il seguente ordine:
\begin{packeditem}
\item maggiore, minore: per indicare il palo di rango maggiore;
\item minore, maggiore: per indicare il palo di rango minore.
\end{packeditem}
 
Lo scarto di spoglio e faglio si esegue sempre con le tre carte più alte sul compagno e con le tre più basse sugli avversari, per cui l'eventuale quarta non è mai significativa.
    \item[Scarto di abbandono:] 
 da eseguire --- se c'è la possibilità --- solo in caso di assoluta inefficienza in tutti i pali, va effettuato diligentemente al fine di non creare nel compagno false supposizioni di forza.
\end{description}
\subsection{Accettazione e rifiuto sulle giocate del compagno}
Si è invitati a manifestare accettazione o rifiuto del palo di giocata del compagno solo quando questi, dichiarando il palo di giocata:
\begin{packeditem}
\item       batte il 3 del 29;
\item       batte il 2 del 28 secco o terzo;
\item       liscia un 3 o un 2;
\item       batte un 25 terzo o lungo.
\end{packeditem}
Se, invece, il compagno gioca in silenzio, e in tutti gli altri casi, la carta o le carte di risposta non significano né accettazione nè rifiuto.
\begin{description}
    \item[Accettazione:]
il giocatore che, sul 3 del 29 o sul 3 lisciante, o sul 2 del 28 o sul 2 lisciante del compagno, risponde con una cartina dello stesso palo o che, su due sue battute sovrane,  risponde con due carte in ordine crescente, manifesta accettazione del gioco ed invita il compagno a proseguire il gioco allo stesso palo.
    \item[Rifiuto:] 
il giocatore che, nelle medesime circostanze, risponde con una figura dello stesso palo o con due carte in ordine decrescente, manifesta rifiuto del gioco ed invita il compagno a non proseguire il gioco allo stesso palo.
\end{description}
\section{I segnali scientifici}
In ogni passata il giocatore esegue dieci giocate ma la decima è, ovviamente, una giocata obbligata.
Le nove giocate libere sono costituite, in media, per il 25\% da battute, per un altro 25\% da risposte a battute del compagno e per il 50\% da risposte a battute degli avversari.
Sulle battute del compagno e degli avversari il giocatore, statistica-mente, risponde all'incirca nel 60\% e scarta nel 40\% dei casi.
Pertanto, delle nove carte effettivamente giocate ogni giocatore in media ne batte 2.25, ne scarta 2.7 e risponde con 4.05.
La risposta al palo è, quindi, nell'arco di una serie di partite, l'atto di gioco più frequente eppure è quello che solitamente viene fatto con maggiore indifferenza; sulla battuta vincente di un avversario ci si limita a fornire la cassa più bassa o ad eseguire lo scarto indiretto; su quella del compagno, invece, la carta più bassa o la più alta per indicare rispettivamente, come abbiamo già visto, accettazione o rifiuto. La possibilità di eseguire lo scarto diretto o quello indiretto,copre al massimo, sempre statisticamente, una delle quattro carte medie di risposta.

In definitiva, quindi, più di un terzo delle giocate complessive di ciascun giocatore sono eseguite in maniera casuale e non incidono, come potrebbero, sull'andamento della passata.

Utilizzando la tecnica dei controscarti fissi sarà possibile, senza nulla togliere agli altri strumenti segnaletici, ridurre di un terzo questo spreco:  il che consente scherzosamente di dire che chi si avvale della tecnica dei controscarti gioca con una carta in più!
 
Se il giocatore non è faglio al palo di giocata non ha l'opportunità di scartare, una o più volte, per poter chiarire, con minore o maggiore precisione, il proprio gioco al compagno. Tale situazione può presentarsi sia su giocate sovrane del compagno sia su giocate sovrane dell'avversario e, in entrambi casi, può accadere che entrambi i giocatori della coppia non possano per due o più battute scambiarsi alcuna informazione proprio mentre almeno uno degli avversari, se non entrambi, ha la possibilità scartando di chiarire il gioco al suo compagno.

Per la coppia, pertanto, sarebbe estremamente positivo potersi scambiare informazioni anche quando si è costretti a rispondere più di una volta al palo di battuta e ciò, in modo particolare, quando si gioca senza accusare il buongioco e sulle prime battute della passata.
A tale esigenza soccorre la tecnica dei controscarti che può essere adoperata, fuori dei casi previsti per le manifestazioni di accettazione o rifiuto sulla giocate del compagno, quando si è costretti a rispondere al palo.

Per rendere possibile il controscarto bisogna che sia stato dato un valore convenzionale alle carte di risposta al palo di battuta, valore che, si badi bene, deve però essere valorizzato dal compagno solo quando le stesse carte di risposta avrebbero potuto essere rilasciate in maniera casuale o, comunque, non significativa.

 Tale tecnica, che --- come vedremo --- richiama nell'articolazione
 quella degli scarti ma che a questa non è alternativa in quanto --- come
 abbiamo già visto --- può essere utilizzata solo quando quella non è
 utilizzabile, prende il nome di tecnica della \emph{significatività delle
 risposte indifferenti} o, più semplicemente, \emph{dei controscarti}.
Le due tecniche, spesso, possono essere utilizzate contemporaneamente e, quindi, l'una può essere contrapposta all'altra: chi è faglio ``scarta''; chi può, e deve, rispondere, ``controscarta''. Di conseguenza, il gioco diventa meno casuale e, quindi, ancora più interessante.
La tecnica dei controscarti si fonda sulla \emph{convenzione primaria} del seguente valore decrescente di rango dei quattro pali: denari, spade, coppe, bastoni.

Escluso il palo di giocata, tra gli altri tre pali esiste sempre una gerarchia univoca: superiore, medio, inferiore.
Poiché anche per le carte del palo di risposta, considerate per il loro
grado, esiste una gerarchia, sarà sempre possibile determinare fra tre
carte, la Superiore, la Media e la Inferiore o fra due carte, la Superiore e la Inferiore.

Per estensione, però, il concetto gerarchico può riferirsi, con evidenti minori certezze, anche al rilascio di una sola carta, che deve essere valutata non in rapporto ad un'altra o ad altre due carte giocate di seguito allo stesso palo, ma per il suo valore assoluto. A tal fine, le tre figure devono essere considerate tendenzialmente superiori, il 7 ed il 6 tendenzialmente medie, il 5 ed il 4 tendenzialmente inferiori.
 
Combinando, quindi, il rango dei pali con il valore (relativo o assoluto) delle carte, è possibile costruire il sistema dei controscarti fissi, detto così per distinguerlo dal sistema dei controscarti variabili.

Tale sistema tende a governare l'intreccio tra \emph{la situazione del palo
di battuta e di risposta} e le possibili \emph{situazioni di interesse} nei pali diversi da quello di battuta.  

\`E evidente che, qualunque sia la situazione del palo di battuta e di risposta, in mancanza di altre informazioni già acquisite, non è possibile chiarire completamente tutte le situazioni in cui possono trovarsi gli altri tre pali e, pertanto, i controscarti, come del resto gli scarti, non sempre forniscono certezze assolute. 

Sovviene a questo punto la possibilità di utilizzare una proprietà
insita nella convenzione primaria (denari > spade > coppe > bastoni) per
la quale il palo di denari (sempre Superiore) e quello di Bastoni
(sempre Inferiore) sarebbero \emph{univoci} mentre gli altri due
(spade e coppe) sarebbero \emph{equivoci}. Combinando tale proprietà con
una \emph{convenzione secondaria} per la quale ci si riferisce sempre al
palo alto (cioè al palo di denari o, se denari è il palo di battuta, a
quello di spade), in termini affermativi rispondendo I-S e in termini escludenti rispondendo S-I, avremo sempre assoluta certezza quando l'interesse è (almeno) per il palo di denari o (almeno) per il palo di spade, mentre avremo incertezza quando l'interesse è per il palo di coppe o per il palo di Bastoni o per entrambi. Pertanto:
\begin{table}[h]
    \centering
    \begin{tabular}{*2c l}
        battuta & risposte & è certo\\\hline
        \multirow{2}{*}{denari} & I-S & interesse a spade\\
        & S-I & non-interesse a spade\\    \hline
        \multirow{2}{*}{spade} & I-S & interesse a denari\\
        & S-I & non-interesse a denari\\        \hline
        \multirow{2}{*}{coppe} & I-S & interesse a denari\\
        & S-I & non-interesse a denari\\             \hline
        \multirow{2}{*}{bastoni} & I-S & interesse a denari\\
        & S-I & non-interesse a denari\\                  \hline
    \end{tabular}
\end{table}\\
La risposta Inferiore-Superiore prende pertanto il nome di
\emph{controscarto certo}, mentre quella Superiore-Inferiore  prende il
nome di \emph{controscarto incerto}.\footnote{In pratica, si rende assolutamente univoco il palo di denari o, se denari è di battuta, quello di spade, in danno del palo di bastoni che, come quello di coppe, sarà sempre equivoco.

Tale convenzione avvantaggia chi, trovandosi ad esempio nella delicata situazione di dover dichiarare l’andata per cappotto, ha così la possibilità di ottenere informazioni assolutamente precise sul palo che gli interessa purché tale palo sia proprio quello di denari, o quello di spade se denari è il palo di battuta.}

Prima di esporre le regole dei controscarti fissi è necessario delimitare il campo entro il quale tale tecnica può essere applicata.
Abbiamo visto, infatti, che, nell'ambito della tecnica segnaletica convenzionale, oltre alle giocate di scarto sono state codificate altre giocate segnaletiche (queste, di risposta al palo) tendenti anch'esse ad orientare il gioco del compagno attraverso l'accettazione o il rifiuto del palo di giocata.
Inoltre, anche le risposte al palo di giocate sovrane degli avversari, per ovvi motivi, non sempre possono essere effettuate in maniera indifferente.
Poiché la tecnica dei controscarti si fonda proprio sull'opportunità di dare un significato all'ordine con cui sono fornite le carte di risposta al palo, non si può ignorare che in alcuni casi sarebbe impossibile comprendere se, su proprie giocate sovrane, il compagno sta eseguendo la tecnica del controscarto o, piuttosto, una giocata di accettazione o rifiuto, o ancora se, su giocate sovrane degli avversari, stia eseguendo il controscarto o un tentativo di difesa del palo.

I controscarti fissi non devono ovviamente inficiare il grado di certezza delle altre informazioni segnaletiche né, tanto meno, compromettere la possibile difesa di un palo, ma possono essere utilizzate per integrare, quando è possibile, il quadro segnaletico complessivo.
Occorre, pertanto, ribadire che la tecnica dei controscarti fissi va utilizzata con assoluta precisione e, tassativamente, nei soli casi per i quali è stata elaborata.

Non si esclude che la stessa potrebbe risultare utile anche in particolari situazioni diverse ma analoghe a quelle codificate, ma il rischio di scivolare nei controscarti variabili e, quindi, di confondere, piuttosto che chiarire, le idee al compagno è molto elevato e i possibili danni ben più consistenti degli eventuali vantaggi.

Un'altra osservazione importante è la seguente. La difficoltà maggiore non
ricade --- come può sembrare --- su chi, per giunta subitamente, deve eseguire  il controscarto (a tutto si fa l'abitudine!)  bensì sul suo compagno che deve interpretarlo.
Per il primo, infatti, in linea di massima è sufficiente rispondere I-S per affermare e  S-I per escludere l'interesse al palo superiore, o, come vedremo per il controscarto singolo, rispondere con una sola carta di grado corrispondente al rango del palo più forte.

 Il suo compagno, invece, deve dare significato al controscarto del compagno nel contesto generale della smazzata. C'è però da dire, a sollievo di questi, che, una sua erronea interpretazione del controscarto gli farebbe solo perdere una grande opportunità, facendolo ritrovare nella stessa situazione di incertezza nella quale si sarebbe trovato se la coppia avesse scelto di fare a meno di tale tecnica.

 Occorre infatti ripetere che la tecnica dei controscarti, come del resto tutte le tecniche segnaletiche convenzionali, non è obbligatoria, ma se preliminarmente si dichiara di volerla utilizzare, poi, all'occorrenza bisogna utilizzarla e utilizzarla bene. Al riguardo va detto che è anche possibile che solo uno o due giocatori dichiarino di non volerla utilizzare; in tal caso, la tecnica potrà comunque essere utilizzata dalla coppia formata dagli altri due o da due degli altri tre giocatori che intendano invece adoperarla.
Sarebbe scorretto, però, adoperare tale tecnica all'insaputa di uno o due giocatori.
 
\subsection{I controscarti fissi}
Il regolamento del controscarto si compendia in due gruppi di norme. Il primo gruppo sancisce ``come'', il secondo ``se e quando'' si controscarta.
 
Rispetto al ``come'' i controscarti si distinguono in:
\begin{description}
\item[triplice pieno (o full):] quando si deve e si può rispondere almeno tre volte;
\item[triplice misto (o mix):] quando si deve rispondere almeno tre volte e si può rispondere solo due volte (più uno o più scarti);
\item[duplice (o di orientamento):] quando si deve (e si può) rispondere due volte;
\item[flash:] dovendo rispondere ad una singola battuta.
\end{description}
 
 
Il \emph{controscarto full} si effettua rispondendo:
\begin{packeditem}
    \item con le prime due in ordine ascendente (\emph{controscarto forte}) per affermare interesse al palo corrispondente alla terza risposta e ad un altro palo);
    \item con le prime due in ordine discendente (\emph{controscarto
        debole}) per affermare interesse al solo palo corrispondente alla terza risposta)\footnote{Nel controscarto full, le definizioni controscarto forte e controscarto debole sono adoperate per analogia con le modalità seguite nei controscarti mix e di orientamento, ma con significato diverso. Nel controscarto full, infatti, la dichiarazione di forza o di debolezza non si riferisce al palo di rango maggiore ma a quello corrispondente alla terza giocata di risposta.};
\end{packeditem}
 
Il \emph{controscarto mix} si effettua rispondendo:
\begin{packeditem}
\item     due carte in ordine crescente (I-S), per affermare interesse al
    palo Superiore (\emph{controscarto forte}), per poi chiarire
\begin{packeditem}
\item       con lo scarto del palo medio, l'interesse anche al palo inferiore;
\item      con lo scarto del palo inferiore, l'interesse anche al palo medio;
\item       con lo scarto del palo superiore, l'interesse anche agli altri due pali.
\end{packeditem}
 
\item     due carte in ordine decrescente (S-I), per escludere
    interesse al palo Superiore (\emph{controscarto debole}), per poi chiarire
\begin{packeditem}
\item       con lo scarto del palo medio, l'interesse al palo medio;
\item      con lo scarto del palo inferiore, l'interesse al palo inferiore;
\item       con lo scarto del palo superiore, l'interesse a nessun palo.
\end{packeditem}
\end{packeditem}
 
Il \emph{controscarto di orientamento} si effettua rispondendo:
\begin{packeditem}
\item Inferiore-Superiore, per affermare interesse al palo Superiore
    (\emph{controscarto certo}).
\item Superiore-Inferiore, per escludere interesse al palo Superiore
    (\emph{controscarto incerto});
\end{packeditem}
 
Il \emph{controscarto flash} --- che è sempre affermativo --- si effettua rispondendo con la carta corrispondente al palo cui si è interessati.
 
I controscarti passivi del tipo duplice (di orientamento) e triplice misto (mix) si effettuano sempre utilizzando le due carte più basse.
 
Nel caso scarti e controscarti di compagno e avversari forniscano segnalazioni contrastanti, occorre rispettare la seguente gerarchia:
\begin{packeditem}
\item controscarto triplice del compagno (full o mix);
\item scarto esplicito del compagno;
\item controscarto duplice certo del compagno;
\item scarto implicito del compagno;
\item controscarto duplice incerto del compagno;
\item controscarto flash del compagno;
\item scarto esplicito o implicito dell'avversario;
\item controscarto triplice, duplice o flash dell'avversario.
\end{packeditem}
 
Rispetto al ``se e quando'' eseguire il controscarto, occorre
distinguere i controscarti ``attivi'' da quelli ``passivi'', a seconda che
si eseguano sulle battute del compagno o su quelle degli
avversari.

Si inizia ad eseguire il controscarto solo quando il Compagno (o l'Avversario) dichiara esplicitamente lo stato del palo al quale gioca e lo stato coincide con uno dei seguenti:
 
Compagno:
\begin{packeditem}
\item         battuta iniziale di Asso di una napoletana;
\item         battuta iniziale di 2 del 25 piombo o lunghissimo;
\item         battuta di 3 piombo;
\item         battuta iniziale di 2 o di Re di un 28 almeno lungo;
\item         battuta iniziale di Re del liscio e busso;
\item         battuta iniziale di due carte divenute entrambe notoriamente sovrane. 
\end{packeditem}
 
Avversario:
\begin{packeditem}
\item         qualsiasi battuta di una napoletana;
\item         battuta iniziale di 2 del 25 piombo o lunghissimo;
\item         battuta di 3 piombo;
\item         battuta iniziale di due o tre carte tutte divenute sovrane.
\end{packeditem}
\subsection{Approfondimenti sul controscarto}
\subsubsection{25 piombo}
Si premette che al giocatore in possesso di un 25 piombo si presentano le seguenti possibilità di gioco:
\begin{packeditem}
\item battere in silenzio un pezzo del 25 e poi giocare ad un altro palo;
\item battere in silenzio prima un pezzo e poi l'altro pezzo del 25;
\item dichiarare il 25 piombo e quindi giocarne, consecutivamente, entrambi i pezzi (preferibilmente, per correttezza, prima il 2 e poi il 3)
\end{packeditem}
e che solo nel terzo caso, ed avendo almeno due carte al palo, le risposte del compagno si configurano come controscarti.

Qualunque sia il numero (superiore ad uno) delle carte che si posseggono al palo, le risposte potranno essere evidentemente solo due: prima la Superiore e poi l'Inferiore o viceversa.
Il giocatore che risponde è libero, avendo tre o più carte, di destinare alla duplice risposta le due carte che ritiene più opportuno. Ad esempio, possedendo l'Asso, potrà decidere di metterlo al sicuro o di trattenerlo per ricavarne una presa una volta rilevata la mano.

Dovrà, però, fare attenzione all'ordine con cui giocherà le due carte prescelte. In particolare, giocando:
\begin{description}
    \item[Inferiore-Superiore (controscarto certo):] indicherà di avere certamente interesse al palo di rango superiore, senza escludere interesse anche ad uno degli altri due pali (o ad entrambi);
\item[Superiore-Inferiore (controscarto incerto):] indicherà di non avere interesse al palo di rango superiore e, quindi, di averlo ad uno dei due pali di rango inferiore o ad entrambi.
\end{description}
 
In conclusione, scartando l' ipotesi che il giocatore che risponde non abbia interesse ad alcun palo perché tutti parimenti forti o deboli, le situazioni che possono verificarsi sono 6: tre si riferiscono all'interesse ad uno solo dei tre pali e tre all'interesse a due dei tre pali.

La tecnica del controscarto consente in questo caso di orientare la battuta successiva alle due del 25 con possibilità di successo al 100\% in quattro casi e al 50\% in due casi. Le possibilità complessive sono, quindi, statisticamente pari all'incirca all'85\% mentre, non utilizzando la tecnica del controscarto, ed affidandosi al caso, sarebbero pari esattamente al 50\%.
\subsubsection{25 lunghissimo}
Il compagno di chi batte il primo pezzo del 25 lunghissimo
\emph{di Re}
 \begin{packeditem}
     \item se ha l'Asso, deve sempre rilasciarlo subito e, quindi, eseguire eventualmente il controscarto;
     \item se non ha l'Asso, deve eseguire subito il controscarto.
 \end{packeditem}
Il compagno di chi batte il primo pezzo del 25 lunghissimo
\begin{packeditem}
    \item  se ha due carte, deve eseguire il controscarto;
    \item se ha tre carte, deve rilasciare subito la superiore e, quindi, eseguire il controscarto solo se eseguendolo non blocca il palo;
    \item se ha un lungo o lunghissimo, deve eseguire il controscarto sulle prime due battute utilizzando le due carte superiori; quindi risponderà assecondando il gioco del compagno.
\end{packeditem}
\subsubsection{29 lunghissimo}
Il compagno di chi, \emph{battendo il 3}, dichiara ``lunghissimo, la buona''
\begin{packedenum}
    \item se in possesso di due carte di risposta, cioè
        \begin{packeditem}
            \item  del 2 lisciante, deve rilasciarlo senza poter effettuare il controscarto;
            \item  di due carte semplici, deve iniziare il
                regolare controscarto sulla battuta di 3 del
                compagno, salvo il caso che --- essendo di
                quarta mano e non essendo caduto il 2 --- ci sia la possibilità di fare ``il  punto'' rilasciando subito l'unica figura che si possiede.
        \end{packeditem}
    \item se in possesso di tre o più carte di risposta, deve sempre effettuare il controscarto. In particolare:
        \begin{packeditem}
            \item   se possiede il 2, deve giocarlo sul 3 e, quindi:
                \begin{packedenum}
                    \item   se possiede solo altre due carte, le deve scartare come quando deve rispondere solo due volte alla Napoletana del compagno;
                    \item se possiede ancora più di due carte, le deve giocare come quando risponde tre volte alla napoletana del compagno;
                \end{packedenum}
            \item  se non possiede il 2, deve iniziare, ed eventualmente proseguire, il controscarto come nel caso della Napoletana del compagno.
        \end{packeditem}
\end{packedenum}

Il compagno di chi, \emph{giocando una carta semplice}, dichiara ``lunghissimo, la buona''
\begin{packedenum}
    \item se in possesso di due carte di risposta, cioè
        \begin{packeditem}
        \item del 2 lisciante, deve prendere regolarmente con il 2 e tornare al compagno;
        \item di due carte semplici, deve iniziare il regolare controscarto solo se ciò non comporta il rischio che gli avversari conseguano la presa senza giocare il 2.
        \end{packeditem}
    \item se in possesso di tre carte di risposta, cioè
        \begin{packeditem}
        \item del 2 terzo, deve prendere regolarmente con il 2 e tornare al compagno con l'inferiore (in silenzio con la superiore lisciando) con lo stesso ordine visto per il caso di duplice risposta al palo della Napoletana del compagno;
        \item di tre carte semplici, deve iniziare il regolare controscarto solo se ciò non comporta il rischio che gli avversari conseguano la presa senza giocare il 2.
        \item di quattro carte, deve sempre scartare la carta più piccola per poi, quando il compagno rientrerà in gioco, effettuare il controscarto con le altre tre carte.
        \end{packeditem}
\end{packedenum}
\subsubsection{Re del liscio e busso}
Su una battuta di Re di un liscio e busso (terzo), il compagno privo di carte di tressette se siede alla sua destra, effettua il controscarto flash; se siede di fronte o alla sua sinistra, effettua il controscarto flash se l'avversario o gli avversari non hanno giocato una carta di tressette.
\subsubsection{25 dell'avversario}
Sul 25 piombo dell'avversario si esegue il controscarto di orientamento con le due carte più basse.
 
Sul 25 terzo dell'avversario non si esegue il controscarto.
 
Sul 25 lungo o lunghissimo dell'avversario si esegue sempre il controscarto di orientamento utilizzando le due carte più basse ma, avendo l'Asso lisciante, questo non deve mai essere rilasciato sulla prima battuta.
Se, sul 25 almeno lungo, le carte di risposta sono due senza l'Asso, sull'eventuale terza battuta sovrana dell'avversario si scarta trasformando il controscarto di orientamento in controscarto mix.
 
Sul 25 dell'avversario, quindi, anche disponendo di tre o più carte di risposta,  non va mai eseguito il controscarto full.
\subsection{La Napoletana convenzionale}
La napoletana può essere battuta in maniera casuale e, quindi, senza dare
alcun significato all'ordine con cui le tre carte  di tressette (e le altre
eventuali) sono giocate, oppure in \emph{maniera convenzionale}.
 
Se si gioca senza napoletana convenzionale il compagno di chi batte la
napoletana eseguirà senz'altro il controscarto triplice
(\emph{full} o \emph{mix}) o, disponendo di una sola carta di risposta, eseguirà con la seconda e con la terza risposta un regolare scarto diretto o di eliminazione.
 
Se invece si gioca con la napoletana convenzionale, il giocatore che si accinge a battere la napoletana può scegliere tra due alternative:
\begin{packeditem}
    \item può interrogare direttamente il compagno sul palo di rango medio;
    \item può lasciare al compagno la possibilità di eseguire il normale controscarto triplice.
\end{packeditem}
Nel primo caso chi batte la napoletana gioca le carte nella prima
colonna, dichiarando quindi quanto riportato nell'ultima, riferendosi
sempre agli altri due pali (cioè n\'e quello medio n\'e quello della
napoletana):
\begin{table}[h]
    \centering
    \begin{tabular}{ccl}
        3 2 A & interroga sul palo medio &  possiede almeno un asso\\
        3 A 2 & senza altri interessi & non possiede nemmeno un asso\\\hline
        2 3 A & interroga sul palo medio  &possiede almeno un due nel palo superiore\\
        2 A 3 & ed è interessato a un altro palo & possiede almeno un due nel palo inferiore\\
    \end{tabular}
\end{table}\\
Per il compagno in possesso di \emph{tre carte di risposta}, le possibili risposte alle interrogative sul palo medio senza altri interessi, cioè alla napoletana del compagno iniziata di 3, saranno:
\begin{description}
    \item[risposta forzante (ISM):] 2 o 3 almeno terzo al palo medio;
    \item[risposta positiva (IMS):] 2 lisciante o 3 lisciante al palo
        medio;
    \item[risposta negativa (SIM):] 2 piombo o 3 piombo al palo medio;
    \item[risposta frustrante (SMI):] esclude il 2 o il 3 al palo medio.
\end{description}
Mentre le possibili risposte alle interrogative sul palo medio con interesse in un altro palo, cioè alla napoletana del compagno iniziata di 2,  saranno le seguenti:
\begin{description}
    \item[risposta forzante (ISM):] 2 o 3 non piombi al palo medio e al
        secondo palo di interesse;
    \item[risposta positiva (IMS):] 2 o 3 non piombi al palo medio;
    \item[risposta negativa (SIM):]  2 o 3 non piombi al palo
        d'interesse;
    \item[risposta frustrante (SMI):] esclude almeno un 2 o un 3
        lisciante ad entrambi i pali.
\end{description}

Per il compagno in possesso di \emph{due carte} le risposte saranno: I-S (risposta positiva) per dichiarare o S-I (risposta negativa) per escludere il possesso di un 2 o di un 3 almeno liscianti al palo medio.
Sulla terza battuta, quindi, scarterà al palo più debole dopo la
risposta positiva, e al palo più forte dopo quella negativa.

Nel caso di invito al normale controscarto chi gioca la napoletana batterà innanzitutto l'Asso per chiedere al compagno di eseguire il controscarto e poi:
\begin{table}[h]
    \centering
    \begin{tabular}{cl}
        A 2 3 &  possiede almeno un due\\
        A 3 2 &  non possiede nemmeno un due\\
    \end{tabular}
\end{table}\\
Il compagno, quindi, eseguirà, a seconda dei casi, il controscarto triplice (full o mix) o il regolare scarto.

Pertanto, il giocatore che ha battuto la Napoletana Convenzionale, sia che
abbia ottenuto dal compagno tre carte di risposta (controscarto full), sia
che ne abbia ottenuto due ed una di scarto (controscarto mix), sia,
ancora, che ne abbia ottenuto una seguita da uno scarto diretto o da uno
scarto di eliminazione, potrà eseguire la battuta successiva alla
Napoletana con garanzia di successo al 90\%.
 
In conclusione, appare evidente che, in mano a giocatori esperti, la Napoletana Convenzionale con le relative risposte o il controscarto sulla Napoletana Non Convenzionale possono diventare un'arma davvero micidiale per gli avversari!
 
Resta da chiedersi (se si gioca con la Napoletana Convenzionale) quando al giocatore in possesso della napoletana convenga fare l'interrogazione al palo medio e quando, invece, (iniziando a battere o prendendo di Asso) invitare il compagno ad eseguire il controscarto.
 
La scelta dipende da una serie di circostanze. Ad esempio, se il giocatore che batte la napoletana ha interesse solo al palo medio, gli converrà interrogare il compagno su tale palo mentre, se la sua forza è distribuita o non ha alcuna forza, può convenirgli invitarlo al controscarto.
La scelta, però, può dipendere da una serie di altre circostanze quali: la
conoscenza già acquisita di altre informazioni (sfide, piombi, etc.), il
punteggio della partita, la possibilità di essere chiamati fuori, o altre
ancora.
\end{document}
